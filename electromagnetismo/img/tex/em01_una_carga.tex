\documentclass[border=3pt]{standalone}

%Drawing
\usepackage{tikz}

%3D
\usepackage{tikz-3dplot}

%Tikz Library
\usetikzlibrary{angles, quotes, intersections}

%Styles
\tikzset{axis/.style={thick,-latex}}
\tikzset{vec/.style={thick,blue, -latex}}
\tikzset{univec/.style={thick,red,-latex}}

%Notation
\usepackage{physics}
\usepackage{bm}

\begin{document}

	\tdplotsetmaincoords{70}{110}

	\pgfmathsetmacro{\sca}{2}

	%
	% posicion q:
	\pgfmathsetmacro{\qx}{2.5*\sca}
	\pgfmathsetmacro{\qy}{2*\sca}
	\pgfmathsetmacro{\qz}{2.9*\sca}
	\pgfmathsetmacro{\px}{1*\sca}
	\pgfmathsetmacro{\py}{3*\sca}
	\pgfmathsetmacro{\pz}{3.25*\sca}
	% \pgfmathsetmacro{\qtheta}{48.17}
	% \pgfmathsetmacro{\qphi}{63.5}
	%

	\begin{tikzpicture}[tdplot_main_coords]
		%Axis
		\draw[axis] (0,0,0) -- (4*\sca,0,0) node [pos=1.05] {$i$};
		\draw[axis] (0,0,0) -- (0,4*\sca,0) node [pos=1.05] {$j$};
		\draw[axis] (0,0,0) -- (0,0,4.5*\sca)  node [pos=1.05] {$k$};

		%Charge
		\draw[dashed] (\qx,0,0) -- (\qx,\qy,0) node [pos=0, above left] {$x_0$};
		\draw[dashed] (0,\qy,0) -- (\qx,\qy,0) node [pos=0, above] {$y_0$};
		\draw[dashed] (0,0,\qz) -- (\qx,\qy,\qz) node [pos=0, left] {$z_0$};
		\draw[vec] (0,0,0) -- (\qx-0.1,\qy-0.1,\qz-0.1) node[pos=0.5, above left] {$\vec{\text{x}}_0$};
		
		%P
		\draw[dashed] (\px,\py,\pz) -- (\px,\py,0);
		\draw[dashed] (\px,0,0) -- (\px,\py,0) node [pos=0, above left] {$x$};
		\draw[dashed] (0,\py,0) -- (\px,\py,0) node [pos=0, above] {$y$};
		\draw[dashed] (0,0,\pz) -- (\px,\py,\pz) node [pos=-0.1] {$z$};
		\draw[vec] (0,0,0) -- (\px-0.1,\py-0.1,\pz-0.1) node[pos=0.75, below] {$\vec{\text{x}}$};
		\fill[white] (\qx,\qy,4.3) circle (4pt);
		\draw[dashed] (\qx,\qy,\qz) -- (\qx,\qy,0);
		
		\draw[vec, red] (\qx,\qy,\qz) -- (\px,\py,\pz) node[pos=0.4, above] {$\vec{\text{r}}$};
		\fill[black] (\qx,\qy,\qz) circle (2.3pt) node [above] {$q$};

		\draw[vec, red] (\px,\py,\pz) -- +(0.4*\px-0.4*\qx, 0.4*\py - 0.4*\qy, 0.4*\pz - 0.4*\qz) node[pos=0.4, above] {$\hat{r}$};

		%Angles
		% \tdplotdrawarc{(0,0,0)}{0.7}{0}{\phivec}{below}{$\phi$}

	\end{tikzpicture}

\end{document}