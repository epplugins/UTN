\twocolumn[\colorsection{Transmisión del calor}]
\setcounter{figure}{0}
%
\begin{Exercise}
  \ifthenelse{\equal{\seleccionados}{true}}
  {\addToList{xyz-transmision}{\ExerciseHeaderNB}}{}
  Un extremo de una varilla metálica aislada se mantiene a $\SI{100}{\celsius}$, y el otro se mantiene a $\SI{0}{\celsius}$ con una mezcla de hielo y agua. La varilla tiene $\SI{60}{\centi\metre}$ de longitud y el área de su sección transversal es $\SI{1.25}{\square\centi\metre}$. El calor conducido por la varilla funde $\SI{8.5}{\gram}$ de hielo cada $\SI{10}{\minute}$. Calcule la conductividad térmica del metal.
\end{Exercise}
\begin{Answer}
  $\SI{227}{\watt.\metre^{-1}.\kelvin^{-1}}$
\end{Answer}
%
\begin{Exercise}
  A través de una ventana de vidrio de $\SI{1}{\square\metre}$ de área y $\SI{5}{\milli\metre}$ de espesor fluye calor a razón de $\SI{1600}{cal/\second}$, siendo la temperatura interior de $\SI{15}{\celsius}$ y la exterior de $\SI{25}{\celsius}$. Si la temperatura exterior aumenta a $\SI{35}{\celsius}$, ¿por cuál de las siguientes ventanas de área $A$ $\si{\square\metre}$ y espesor $w$ $\si{\milli\metre}$ la cambiaría para mantener el mismo flujo calórico?\\
\begin{tabular*}{0.8\textwidth}{ccc}
  \textit{a}) $A/w=0.40$ & \textit{b}) $A/w=0.13$ & \textit{c}) $A/w=0.20$\\
  \textit{d}) $A/w=0.17$ & \textit{e}) $A/w=0.11$ & \textit{f}) $A/w=0.10$\\
\end{tabular*}
\end{Exercise}
\begin{Answer}
  Opción \textit{f})
\end{Answer}
%
\begin{Exercise}
  Un método experimental para medir la conductividad térmica de un material aislante consiste en construir una caja del material y medir el aporte de potencia a un calentador eléctrico dentro de la caja, que mantiene el interior a una temperatura medida por encima de la temperatura de la superficie exterior. Suponga que en un aparato como el mencionado se requiere un aporte de potencia de $\SI{180}{\watt}$ para mantener la superficie interior de la caja $\SI{65.0}{\celsius}$ arriba de la temperatura de la superficie exterior. El área total de la caja es de $\SI{2.18}{\square\metre}$, y el espesor de la pared es de $\SI{3.90}{\centi\metre}$. Calcule la conductividad térmica del material en unidades del SI.
\end{Exercise}
\begin{Answer}
  $\SI{0.0495}{\watt.\metre^{-1}.\kelvin^{-1}}$
\end{Answer}
%
\begin{Exercise}\label{p:transmision00}
  \ifthenelse{\equal{\seleccionados}{true}}
  {\addToList{xyz-transmision}{\ExerciseHeaderNB}}{}
  En una casa se tiene una pared de ladrillos de $\SI{3}{\metre} \times \SI{4}{\metre}$, y espesor $\SI{15}{\centi\metre}$, que separa un ambiente a $\SI{25}{\celsius}$ del exterior a $\SI{5}{\celsius}$. Esta pared contiene una ventana que consiste en solo un panel de vidrio de $\SI{1.5}{\metre} \times \SI{1.5}{\metre} \times \SI{5}{\milli\metre}$. \textit{a}) Calcular la corriente de calor total a través del concreto y la ventana, sin incluir efectos de convección. \textit{b}) ¿Cuál es el porcentaje de calor que se pierde a través de la ventana respecto del total? \textit{Datos}: conductividad térmica del ladrillo = $\SI{0.6}{\watt.\metre^{-1}.\kelvin^{-1}}$; conductividad térmica del vidrio = $\SI{1}{\watt.\metre^{-1}.\kelvin^{-1}}$.
\end{Exercise}
\begin{Answer}
	\begin{minipage}[t]{.4\textwidth}
    \textit{a}) $\SI{9780}{\watt}$\\ \textit{b}) 92\%
  \end{minipage}
\end{Answer}
%
%No se informan las conductividades, con la intención de que los estudiantes las busquen.
\begin{Exercise}
  \ifthenelse{\equal{\seleccionados}{true}}
  {\addToList{xyz-transmision}{\ExerciseHeaderNB}}{}
  Dos barras, una de latón y otra de cobre, están unidas extremo con extremo. La longitud de la barra de latón es $\SI{0.2}{\metre}$ y la de cobre es $\SI{0.8}{\metre}$. La sección transversal de cada segmento tiene un área de $\SI{0.005}{\square\metre}$. El extremo libre del segmento de latón está en contacto con agua hirviendo y el extremo libre del segmento de cobre se encuentra en contacto con una mezcla de hielo y agua, en ambos casos a presión atmosférica normal. Los lados de las varillas están aislados, por lo que no hay pérdida de calor a los alrededores. \textit{a}) ¿Cuál es la temperatura del punto en el que los segmentos de latón y de cobre se unen? \textit{b}) ¿Qué masa de hielo se funde en $\SI{5}{\minute}$ por el calor conducido por la varilla compuesta?
\end{Exercise}
\begin{Answer}
	\begin{minipage}[t]{.4\textwidth}
    \textit{a}) $\SI{53.1}{\celsius}$; \textit{b}) $\SI{115}{\gram}$
  \end{minipage}
\end{Answer}
%
\begin{Exercise}
  Se sueldan varillas de cobre, latón y acero para formar una Y. El área de la sección transversal de cada varilla es de $\SI{2}{\square\centi\metre}$. El extremo libre de la varilla de cobre se mantiene a $\SI{100}{\celsius}$, y los extremos libres de las varillas de latón y acero a $\SI{0}{\celsius}$. Suponga que no hay pérdida de calor por los laterales de las varillas, cuyas longitudes son: $\SI{13}{\centi\metre}$, $\SI{18}{\centi\metre}$ y $\SI{24}{\centi\metre}$ para la de cobre, latón y acero respectivamente. \textit{a}) ¿Qué temperatura tiene el punto de unión? \textit{b}) Calcule la corriente de calor en cada una de las tres varillas.
\end{Exercise}
\begin{Answer}
	\begin{minipage}[t]{.4\textwidth}
    \textit{a}) $\SI{78.4}{\celsius}$\\ \textit{b}) $H_\text{latón} = \SI{9.50}{\watt}$, $H_\text{acero} = \SI{3.28}{\watt}$ y $H_\text{cobre} = \SI{12.8}{\watt}$
  \end{minipage}
\end{Answer}
%
\begin{Exercise}\label{p:transmision01}
  Una pared exterior está compuesta por una capa externa de madera de $\SI{3.0}{\centi\metre}$ de espesor y una capa interna de espuma de poliestireno de $\SI{2.2}{\centi\metre}$ de espesor. Considere que la conductividad térmica de la madera es $k_\text{m} = \SI{0.08}{\watt.\metre^{-1}.\kelvin^{-1}}$ y la del poliestireno es $k_\text{p} = \SI{0.01}{\watt.\metre^{-1}.\kelvin^{-1}}$. La temperatura del aire en el interior es $\SI{19}{\celsius}$ y la del aire en el exterior es $\SI{-10}{\celsius}$, y los coeficientes de convección del aire en el interior y del aire en el exterior valen $\SI{5}{\watt/(\square\metre.\kelvin)}$ y $\SI{12}{\watt/(\square\metre.\kelvin)}$ respectivamente. \textit{a}) Calcule la rapidez del flujo de calor por metro cuadrado a través de esta pared. \textit{b}) Calcule la temperatura en la superficie de contacto entre la madera y la espuma de poliestireno. 
\end{Exercise}
\begin{Answer}
	\begin{minipage}[t]{.4\textwidth}
    \textit{a}) $\SI{10.2}{\watt/\square\metre}$\\ \textit{b}) $\SI{-5.36}{\celsius}$
  \end{minipage}
\end{Answer}
%
\begin{Exercise}
  Un carpintero construye una cabaña rústica que tiene un piso de dimensiones $\SI{3.50}{\metre} \times \SI{3.00}{\metre}$. Sus paredes, que miden $\SI{2.50}{\metre}$ de alto y $\SI{1.80}{\centi\metre}$ de grosor, están hechas de una madera cuya conductividad térmica vale $\SI{0.517}{cal/(\hour.\centi\metre.\celsius)}$, y serán aisladas con un material sintético de conductividad térmica igual a $\SI{0.947}{cal/(\hour.\centi\metre.\celsius)}$. Se desea instalar una estufa que entregue una potencia calorífica de $\SI{1100}{kcal/\hour}$ para mantener el interior a una temperatura de $\SI{25.0}{\celsius}$ cuando la temperatura exterior es $\SI{2.00}{\celsius}$. Despreciando la pérdida de calor a través del techo y del piso, calcule el espesor mínimo necesario del material aislante. Considere que el  coeficiente de convección del aire $\SI{2.00E-4}{cal/(\second.\centi\square\metre.\celsius)}$ tanto en el interior como en el exterior.
\end{Exercise}
\begin{Answer}
  $\SI{0.51}{\centi\metre}$
\end{Answer}
%
\begin{Exercise}
  Calcule la tasa de radiación de energía por unidad de área de un cuerpo negro a: \textit{a}) $\SI{273}{\kelvin}$ y \textit{b}) $\SI{2730}{\kelvin}$.
\end{Exercise}
\begin{Answer}
	\begin{minipage}[t]{.4\textwidth}
    \textit{a}) $\SI{315}{\watt/\square\metre}$\\ \textit{b}) $\SI{3.15E6}{\watt/\square\metre}$
  \end{minipage}
\end{Answer}
%
\begin{Exercise}
  \ifthenelse{\equal{\seleccionados}{true}}
  {\addToList{xyz-transmision}{\ExerciseHeaderNB}}{}
  La emisividad del tungsteno es $0.350$. Una esfera de tungsteno con un radio de $\SI{1.5}{\centi\metre}$ se suspende dentro de una cavidad grande, cuyas paredes están a $\SI{290}{\kelvin}$. ¿Qué aporte de potencia se requiere para mantener la esfera a una temperatura de $\SI{3000}{\kelvin}$, si se desprecia la conducción de calor por los soportes?
\end{Exercise}
\begin{Answer}
  $\SI{4540}{\watt}$
\end{Answer}
%
\begin{Exercise}
  La tasa de energía radiante que llega del Sol a la atmósfera superior de la Tierra es cercana a $\SI{1.5}{\kilo\watt/\square\metre}$. La distancia promedio de la Tierra al Sol es $\SI{1.5E11}{\metre}$ y el radio del Sol es $\SI{6.96E8}{\metre}$. \textit{a}) Calcule la tasa de radiación de energía por unidad de área de la superficie solar. \textit{b}) Si el Sol irradia como cuerpo negro ideal, ¿qué temperatura tiene en su superficie?
\end{Exercise}
\begin{Answer}
	\begin{minipage}[t]{.4\textwidth}
    \textit{a}) $\approx\SI{70}{\mega\watt/\square\metre}$\\ \textit{b}) $\approx\SI{5900}{\kelvin}$
  \end{minipage}
\end{Answer}
%