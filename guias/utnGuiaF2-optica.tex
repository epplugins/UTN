\documentclass[a4paper,11pt,dvipsnames]{article}
\usepackage[a4paper,left=2cm,right=2cm,top=2.5cm,bottom=2.5cm]{geometry}

%%
% Con espacio para anotar:
% \usepackage[a4paper,paperwidth=847pt, textheight=650pt, textwidth=418pt,top=2.5cm,bottom=2.5cm]{geometry}
% \setlength{\oddsidemargin}{17pt}
% \setlength{\marginparwidth}{250pt}
%%

% \usepackage{ucs}
% \usepackage[utf8x]{inputenc}
\usepackage[utf8]{inputenc}
\usepackage[T1]{fontenc}
\usepackage[spanish]{babel}
\usepackage{mathrsfs}
\usepackage{amssymb,amsmath}
\usepackage{enumerate}
% \usepackage{capt-of}
\usepackage{multicol}
\usepackage{caption}
\usepackage[stable]{footmisc} %para footnote dentro de pma
\usepackage{graphicx}
\usepackage{ifthen}
\usepackage{pgfplots}
% \usepackage{pdfpages}
% \usepackage{tikz}
\usepackage{circuitikz}
\usepackage{tikz-3dplot}
% \usepackage{gnuplottex}
\usepackage{xcolor}
\usepackage[locale=FR, per-mode=fraction, separate-uncertainty=true]{siunitx}
\usepackage{physics}
\ExplSyntaxOn
\msg_redirect_name:nnn { siunitx } { physics-pkg } { none }
\ExplSyntaxOff
\usepackage{subfigure}

\input{./pstriks_on_pdf.code.tex}
\usetikzlibrary{decorations}
\usetikzlibrary{decorations.pathmorphing}
\usetikzlibrary{shapes.geometric}
\usetikzlibrary{arrows}
\usetikzlibrary{arrows.meta}   %edg
\usetikzlibrary{calc}
\usetikzlibrary{patterns}
\usetikzlibrary{3d}  %babinet

\pgfplotsset{compat=newest}



%%%%%%%%%%%%%%%%%%%%%%%%%%%%%%%%%%%%%%%%%%%%%%%%
% PORTADA 1ER O 2DO PARCIAL
%%%%%%%%%%%%%%%%%%%%%%%%%%%%%%%%%%%%%%%%%%%%%%%%
% Esto lo dejo por las dudas, cuando analice si se usa lo podré eliminar.
% \newcommand{\parcial1}{true}
% \newcommand{\primerparcial}{true}
\newcommand{\primerparcial}{false}

% Para generar un documento independiente con la unidad 2
% en lugar de incluirlo a continuacion dentro de parcial 1.
% Para compartir con la cátedra.
\newcommand{\unidaddos}{true}

\newcommand{\anio}{2024}
\newcommand{\comision}{2\textsuperscript{do} 31}

% Imprimir lista de ejercicios seleccionados para hacer en clase.
\newcommand{\seleccionados}{false}


\renewcommand{\thefigure}{\arabic{section}.\arabic{figure}}
\renewcommand{\theequation}{\arabic{section}.\arabic{equation}}

\renewcommand\spanishtablename{Tabla}


%---------------------------
%
% Section headers with colors.
%
%---------------------------
% \usepackage{stackengine}
% \usepackage[most]{tcolorbox}

\definecolor{topcolor}{RGB}{0,121,138}


% \newcommand{\colorofsection}{cyan!90!white}
\newcommand{\colorofsection}{topcolor}
\newcommand{\colorsection}[1]{
    \tcbset{on line,
        boxsep=5pt, left=0pt,right=0pt,top=0pt,bottom=0pt,
        colframe=white, colback=\colorofsection, sharp corners=southwest,
        leftrule=0pt, highlight math style={enhanced}
        }

    \refstepcounter{section}%
    \bigskip\bigskip
    {\noindent\def\stackalignment{l}%
    \stackunder[-2pt]{\tcbox{\textcolor{white}{\textbf{\Large\thesection.\hspace{5pt}#1}}}}{\textcolor{\colorofsection}{\rule{\linewidth}{1pt}}}\medskip}
    \addcontentsline{toc}{section}{\thesection\hspace{5pt}#1}
}

%---------------------------
%
% Exercises and answers.
%
%---------------------------
% \usepackage[lastexercise,answerdelayed]{exercise}
% \usepackage{multicol}
\counterwithin{Exercise}{section}
\renewcommand{\ExerciseHeader}{
    \noindent\textbf{\large
    \ExerciseHeaderNB\ExerciseHeaderTitle
    \ExerciseHeaderOrigin}}
\renewcommand{\AnswerHeader}{\noindent\medskip{\textbf{\ExerciseHeaderNB\hspace{5pt}}}}

%---------------------------
%
% Tikz
%
%---------------------------

\tikzset{axis/.style={blue, thick,-latex}}

% Optica:
% \newcommand\planemirror{} % just for safety
\def\planemirror[#1](#2)(#3)(#4)(#5){%
  % Synopsis
  % \planemirror[fill options](center)(length)(angle)(thickness)
  \fill[#1]   (#2) + ({-0.5*#3*cos(#4)},{-0.5*#3*sin(#4)})
  --++ ({0.5*#3*cos(#4)},{0.5*#3*sin(#4)})
  --++ ({#5*sin(#4)},{-#5*cos(#4)}) --++ ({-#3*cos(#4)},{-#3*sin(#4)}) -- cycle;
  \draw (#2) + ({-#3*cos(#4)/2},{-#3*sin(#4)/2}) --++ ({0.5*#3*cos(#4)},{0.5*#3*sin(#4)});
}
% \newcommand\ray{} % just for safety
\def\ray(#1)(#2)(#3)(#4){%
  % Synopsis
  % \ray[](starting point)(end point)(position of arrow)(xscale)
  \draw[red, thick] (#1) -- (#2) node[color=red, currarrow, pos=#3, xscale=#4, sloped] {};
}

%---------------------------
%
% Babinet
%
%---------------------------

\newcommand{\object}[1]{%
	\begin{scope}[canvas is xz plane at y=1.2]
		\draw[line join=round, thick, fill=black!40] (#1,-1.2) rectangle (#1+0.1,1.2);
	\end{scope}
	%
	\begin{scope}[canvas is xy plane at z=1.2]
		\draw[line join=round, thick, fill=black!25](#1,-1.2) rectangle (#1+0.1,1.2);
	\end{scope}
	%
	\begin{scope}[canvas is yz plane at x=#1]
		\draw[line join=round, thick, fill=black] (-1.2,-1.2) rectangle (1.2,1.2);
		\draw[line join=round, thick, fill=white] (0,0.3) -- (0.3,-0.15) -- (-0.3,-0.15) -- cycle;
		\draw[line join=round, thick, fill=black!40] (0.04,0.24) -- (-0.1667,-0.07) -- (0.2467,-0.07) -- (0.3,-0.15) -- (-0.3,-0.15) -- (0,0.3) -- cycle;
		\draw[thick] (-0.3,-0.15) -- (-0.1667, -0.07);
	\end{scope}
}
\newcommand{\objecttriangle}[1]{%
	\begin{scope}[canvas is yz plane at x=#1]
		\draw[line join=round, thick, fill=black] (0,0.3) -- (0.3,-0.15) -- (-0.3,-0.15) -- cycle;
		\draw[line join=round, thick, fill=black!40] (0,0.3) -- (0.1333, 0.38) -- (0.4333,-0.07)
		-- (0.3,-0.15) -- cycle;
	\end{scope}
}
\newcommand{\image}[1]{%
	\def\point{0.3}
	\def\inside{0.15}
	\begin{scope}[canvas is yz plane at x=#1]
		\draw[line join=round, thick, fill=black] (-1.2,-1.2) rectangle (1.2,1.2);
		\draw[line join=round, thick, fill=red] (0,\point) -- (\inside*0.5,\inside*0.707) -- (\point*0.707,\point*0.5) -- (\inside,0) -- (\point*0.707, -\point*0.5) -- (\inside*0.5, -\inside*0.707) -- (0,-\point) -- (-\inside*0.5, -\inside*0.707) -- (-\point*0.707, -\point*0.5) -- (-\inside,0) -- (-\point*0.707,\point*0.5) -- (-\inside*0.5,\inside*0.707) -- cycle;
	\end{scope}
}
\newcommand{\lens}[1]{%
	\begin{scope}[canvas is yz plane at x=#1+0.1]
		\draw[cyan!30] (0,0) circle (30pt);
	\end{scope}
	\begin{scope}[canvas is yz plane at x=#1]
		\shade[left color=cyan!0,right color=cyan!30]
		(0,0) circle (30pt);
		\draw[cyan!30] (0,0) circle (30pt);
	\end{scope}
}
%---------------------------


%---------------------------
%
% Genera una lista dinámica con
% ejercicios seleccionados.
%
%---------------------------

\let\svaddtocontents\addtocontents
\makeatletter
\newcommand\defineList[1]{%
 \expandafter\def\csname add#1line\endcsname##1##2##3{\addtocontents {##1}{%
  \protect \csname #1line\endcsname {##2}{##3}}}
 \expandafter\def\csname write#1\endcsname{%
  \renewcommand\addtocontents[2]{\relax}%
  \setcounter{section}{0}\noindent%
  \expandafter\def\csname #1line\endcsname####1####2{\expandafter\csname####1\endcsname{####2}}%
  \@starttoc{#1}%
  \setcounter{section}{0}%
  \let\addtocontents\svaddtocontents%
 }%
 \csname add#1line\endcsname{#1}{begin}{itemize}%
 \AtEndDocument{\csname add#1line\endcsname{#1}{end}{itemize}}
}
\newcommand\addToList[2]{\csname add#1line\endcsname{#1}{item}{#2}}
\newcommand\printList[1]{\csname write#1\endcsname}
\makeatother

%---------------------------

\graphicspath{{./img/}}

% \includeonly{termo_portada.code, termo_dilatacion.code, termo_calorimetria.code,
%   termo_preguntas_calorimetria.code, termo_transmision.code,
%   termo_preguntas_transmision.code, termo_primerppio.code,
%   termo_segundoppio.code, termo_preguntas_principios.code,
%   termo_adicionales.code}

\begin{document}

  \input{./guias_newcommands-tikz.code.tex}

  \setcounter{section}{5}

  % 
\begin{titlepage}
\begin{figure}[ht]
\begin{center}
\vspace{1.5cm}
% Aquí se inserta el escudo o emblema:
\begin{tikzpicture}[scale=1, every node/.style={scale=1}]
    \definecolor{topcolor}{RGB}{255,128,0};
    \definecolor{botcolor}{RGB}{255,128,0};

    \fill[color=topcolor] (1.8273,1.065) arc (30.235:149.765:2.115) -- cycle;
    \fill[color=botcolor] (1.8273,-1.065) arc (-30.235:-149.765:2.115) -- cycle;

    \draw[color=botcolor, thick] (-1.755,-0.625) -- (1.755, -0.625);
    \fill[color=botcolor] (-1.755,-0.0929) -- (1.755, -0.0929) -- (1.755, -0.5405) -- (-1.755,-0.5405) -- cycle;

    % \fill [black](0,-0.7) circle(2pt);

    % \draw (-1.7539,0.93) rectangle (-0.9592,0);
    \fill[] (-1.7539,0.93) arc (180:360:0.39735) -- (-1.1178,0.93) arc (360:180:0.23875) -- cycle;
    \fill[] (-1.7539,0) arc (180:0:0.39735) -- (-1.1178,0) arc (0:180:0.23875) -- cycle;
    \fill (-1.7539,0.3866) rectangle (-0.9592,0.5434);
    \fill (-1.43495,0.93) rectangle (-1.27815,0);

    \fill[color=topcolor] (-0.7795,0.3275) arc[start angle=180, end angle=360,x radius=0.40585, y radius=0.3275] -- (0.0322,0.8506) arc[start angle=0, end angle=180,x radius=0.08925, y radius=0.0794] -- (-0.1463,0.3275) arc[start angle=0, end angle=-180,x radius=0.22735, y radius=0.1604] -- (-0.603, 0.8506) arc[start angle=0, end angle=180,x radius=0.08925, y radius=0.0794] -- cycle;

    \fill[color=topcolor] (0.4003,0.0794) arc[start angle=180, end angle=360,x radius=0.08925, y radius=0.0794] -- (0.5788,0.7515) -- (0.7986,0.7515) arc[start angle=-90, end angle=90,x radius=0.0794, y radius=0.08925] -- (0.1805,0.93) arc[start angle=90, end angle=270,x radius=0.0794, y radius=0.08925] -- (0.4003, 0.7515) -- cycle;

    \fill[color=topcolor] (0.9455,0.0794) arc[start angle=180, end angle=360,x radius=0.08925, y radius=0.0794] -- (1.124,0.608) -- (1.57,0.0267) arc[start angle=-153.217, end angle=0,x radius=0.09411, y radius=0.07059] -- (1.755,0.8506) arc[start angle=0, end angle=180,x radius=0.08925, y radius=0.0794] -- (1.5765,0.3191) -- (1.124,0.9004) arc[start angle=26.783, end angle=180,x radius=0.09411, y radius=0.07059] -- cycle;

    \node [scale=0.8,color=white] at (-1.3822,-0.31) {{\fontfamily{cmss}\selectfont \textbf{A}}};
    \node [scale=0.8,color=white] at (-1.0751,-0.31) {{\fontfamily{cmss}\selectfont \textbf{V}}};
    \node [scale=0.8,color=white] at (-0.768,-0.31) {{\fontfamily{cmss}\selectfont \textbf{E}}};
    \node [scale=0.8,color=white] at (-0.4609,-0.31) {{\fontfamily{cmss}\selectfont \textbf{L}}};
    \node [scale=0.8,color=white] at (-0.1536,-0.31) {{\fontfamily{cmss}\selectfont \textbf{L}}};
    \node [scale=0.8,color=white] at (0.1536,-0.31) {{\fontfamily{cmss}\selectfont \textbf{A}}};
    \node [scale=0.8,color=white] at (0.4605,-0.31) {{\fontfamily{cmss}\selectfont \textbf{N}}};
    \node [scale=0.8,color=white] at (0.7676,-0.31) {{\fontfamily{cmss}\selectfont \textbf{E}}};
    \node [scale=0.8,color=white] at (1.0749,-0.31) {{\fontfamily{cmss}\selectfont \textbf{D}}};
    \node [scale=0.8,color=white] at (1.3822,-0.31) {{\fontfamily{cmss}\selectfont \textbf{A}}};

    \node [scale=1] at (0,-0.845) {{\fontfamily{cmss}\selectfont UDB Física}};

\end{tikzpicture}

% \input{termodinamica/termo_logoFRA}

\vspace{-1cm}
\end{center}
\end{figure}

\begin{center}
{\LARGE UNIVERSIDAD TECNOLÓGICA NACIONAL}\\
\vspace{0.25cm}
{\LARGE Facultad Regional Avellaneda}\\
\vspace{1cm}
{\Huge Física II - \comision}
% {\Huge Física II}

\vspace{1.5cm} {\LARGE Guía de problemas de la unidad II}\\
% \vspace{1.5cm} {\LARGE Guía de problemas}\\
\vspace{1cm}
{\Huge \bf \color[RGB]{255,128,0} Óptica}
% {\Huge \bf \color[RGB]{0,121,138} Óptica}


\vspace*{\fill}
\vspace{0.75cm} {Año \anio}
\vspace{0.75cm}

% \begin{figure}[h]
%   \centering
%   \includegraphics[scale=0.28]{portada1-03.jpg}
%   \hfill
%   \includegraphics[scale=0.28]{portada1-01.jpg}
%     \vspace{3mm}  Luz atravesando un medio material (gotas de agua).
% \end{figure}

\end{center}


\end{titlepage}

\newpage
\pagenumbering{roman}
\tableofcontents

\newpage
\pagenumbering{arabic}

  % \begin{center}
   {\scshape \Huge  Óptica\par}
   \vspace{0.5 cm}
\end{center}

% \ifthenelse{\equal{\unidaddos}{false}}{
% \part{\scshape Óptica}
% \vspace{0.5 cm}
% }

\section{Ondas}
\rfigure
%
\pma{\label{p:P170}
\textit{a}) Ondas de radio de onda media y de frecuencia modulada tienen frecuencias del orden de los $\SI{1200}{kHz}$ y de 120 kHz, respectivamente. Hallar sus
correspondientes longitudes de onda en el vacío. \textit{b}) El oído humano es capaz de percibir sonidos de frecuencias comprendidas entre $\SI{20}{Hz}$ y $\SI{20000}{Hz}$. Determinar las longitudes de onda correspondientes a estas frecuencias, suponiendo que la velocidad del sonido es $\SI{338}{m/s}$. \textit{c}) La radiación de los hornos a microondas o la de las señales Wi-Fi tienen una frecuencia cercana a los $\SI{2.45}{GHz}$. Calcular la longitud de onda de estas radiaciones.\\
\rta{.95}{\textit{a}) Entre $\SI{250}{\metre}$ y $\SI{2500}{\metre}$; \textit{a}) Entre $\SI{0.0169}{\metre}$ y $\SI{16.9}{\metre}$; \textit{c}) $\SI{0.122}{m}$}
}
%
\pma{
Un hombre que se sienta en el borde de un muelle a pescar y cuenta las ondas de agua que golpean un poste de soporte del muelle, en un minuto cuenta 76 ondas. Si una
cresta en particular viaja $\SI{18}{m}$ en $\SI{6}{s}$, ¿cuál es la longitud de onda de estas ondas?\\
\rta{.95}{$\SI{2.37}{m}$}
}
%
\pma{
Una onda armónica en un hilo tiene una amplitud de $\SI{17}{mm}$, una longitud de onda de $\SI{1.96}{m}$ y una velocidad de $\SI{4.1}{m/s}$. Determinar el período, la frecuencia, la frecuencia angular y el número de onda.\\
\rta{.95}{$f = \SI{2.09}{\hertz}$; $T = \SI{0.48}{\second}$; $\omega = \SI{13.13}{\radian/\second}$; $k = \SI{3.2}{\metre^{-1}}$}
}
%
\pma{
La velocidad de las ondas electromagnéticas en el vacío es $\SI{3E8}{m/s}$. Las longitudes de onda de las ondas electromagnéticas visibles se extienden aproximadamente desde $\SI{400}{nm}$ (luz violeta) hasta $\SI{750}{nm}$ (luz roja). Determinar el rango de frecuencias de luz visible.\\
\rta{.95}{Entre $\SI{4.0E14}{\hertz}$ y $\SI{7.5E14}{\hertz}$}
}
%
\pma{
La intensidad promedio de la luz solar sobre la superficie terrestre es de unos $\SI{700}{W/m^2}$. \textit{a}) Calcular la cantidad de energía que incide sobre un panel solar de $\SI{0.5}{m^2}$ de área en 4 horas. \textit{b}) ¿Qué intensidad tendrá la luz solar, si es concentrada por una lupa sobre una superficie 200 veces menor que la propia de la lupa?\\
\rta{.95}{\textit{a}) $\SI{5.04E6}{J}$; \textit{b}) $\SI{1.4E5}{W/m^2}$}
}


  % \section{Superposición coherente entre dos ondas de luz. Interferencia}
\rfigure
%
\pma{
Dos haces de luz linealmente polarizados, con sus campos eléctricos paralelos, se propagan en la misma dirección y sentido. Uno de ellos tiene una intensidad de $\SI{7}{\watt/\metre^2}$ y una frecuencia de 600~THz ($\SI{6E14}{\hertz}$), y el otro una intensidad de $\SI{3}{\watt/\metre^2}$ y una longitud de onda de 500~nm ($\SI{500E-9}{m}$). \textit{a}) Calcule la intensidad resultante si la fase relativa entre ambas ondas se mantiene constante en el tiempo y vale $\varphi=\frac{\pi}{3}$. \textit{b}) Repita el cálculo, si la longitud de onda del segundo haz es 520~nm.\\
\rta{.95}{\textit{a}) Las ondas son coherentes y por lo tanto hay interferencia: $\SI{14.6}{\watt/\metre^2}$; \textit{b}) las ondas son incoherentes, es válida la suma de intensidades: 10 W/m$^2$}}

\pma{
Calcule la mínima diferencia que hay entre los caminos recorridos por dos haces de $600$~nm de longitud de onda en el vacío que al superponerse están en contrafase si: \textit{a}) se propagan en el vacío, \textit{b}) se propagan en agua ($n_{agua}=\frac43$). \textit{Nota}: la luz en ningún caso cambia de medio por el que se propaga ni se refleja y los haces están inicialmente en fase.
\rta{.90}{\textit{a}) $\SI{300}{\nano\metre}$; \textit{b}) $\SI{225}{\nano\metre}$}}
%
\pma{\label{p:PO205}
La figura \ref{f:PO205} muestra dos haces de luz de longitud de onda igual a $\SI{450}{nm}$ que se propagan en aire. En cierta parte del recorrido, el haz 1 atraviesa un objeto traslúcido de longitud $L = \SI{21.4}{\micro\metre}$, con un índice de refracción $n' = 1.45$. \textit{a}) ¿Cuál es la diferencia de fase entre ambos haces luego de que el haz 1 vuelve a propagarse en aire, si inicialmente ambos haces tenían la misma fase? \textit{b}) Encuentre al menos dos valores posibles de $L$ para que los haces estén en contrafase luego de que el haz 1 atraviese el material de índice $n'$.\\
\rta{.90}{\textit{a}) $\SI{134.46}{\radian} = (42\pi+0.4\pi)~\si{\radian}$; \textit{b}) Los distintos valores cumplen $L = (2m-1)\cdot\SI{500}{nm}$ con $m \in \mathbb{N}$}}
%
\pma{\label{p:PO202}
La figura \ref{f:PO202} muestra dos haces de luz de $\SI{650}{nm}$ de longitud de onda que se propagan en aire y cruzan dos capas delgadas de materiales plásticos. El haz 1 atraviesa una distancia $L_1 = \SI{4}{\micro\metre}$ dentro de policarbonato (índice de refracción $n_1 = 1.58$) y el 2 recorre una distancia $L_2 = \SI{3.5}{\micro\metre}$ en acrílico (índice de refracción $n_2 = 1.49$). Considere que ambos haces recorren la misma distancia y que tienen la misma fase antes de ingresar en los plásticos. \textit{a}) ¿Cuál es la diferencia entre los caminos ópticos de cada haz (en unidades de longitud y en múltiplos de la longitud de onda) desde que entran a los plásticos hasta el punto A? \textit{b}) ¿Cuál es la diferencia de fase entre los haces?\\
\rta{.95}{\textit{a}) $\SI{605}{\nano\metre}$ y $0.931\,\lambda_0$; \textit{b}) $\Delta\phi = \SI{5.85}{\radian}$}}
%
\begin{minipage}[c]{0.45\textwidth}
\begin{center}
  \begin{tikzpicture}[line cap=round,line join=round]
  \fill[cyan,opacity=0.3] (0,0) -- (0,2) -- (4,2) -- (4,0) -- cycle;
  \draw (0,0) -- (0,2) -- (4,2) -- (4,0) -- cycle;
  \ray(-1,3)(5,3)(0.5)(1);
  \ray(-1,1)(5,1)(0.5)(1);
  \draw (-0.5,1) node[above] {1};
  \draw (-0.5,3) node[above] {2};
  \draw (1,1.5) node[] {$n'$};
  \draw (0,-0.2) -- (0,-1);
  \draw (4,-0.2) -- (4,-1);
  \draw[latex-latex] (0,-0.7) -- (4,-0.7) node[above, pos=0.5] {$L$};
  \end{tikzpicture}
  \captionof{figure}{Problema \ref{p:PO205}\label{f:PO205}}
\end{center}
\end{minipage}
%
\begin{minipage}[c]{0.45\textwidth}
  \begin{center}
  \begin{tikzpicture}[line cap=round,line join=round]
  \fill[cyan,opacity=0.3] (0,0) -- (0,2) -- (4,2) -- (4,0) -- cycle;
  \fill[orange,opacity=0.3] (0,2) -- (0,4) -- (3,4) -- (3,2) -- cycle;
  \draw (0,0) -- (0,4);
  \draw (0,2) -- (4,2);
  \draw (4,0) -- (4,4) node[above] {$A$};
  \draw (3,2) -- (3,4);
  \ray(-1,3)(5,3)(0.5)(1);
  \ray(-1,1)(5,1)(0.5)(1);
  \draw (-0.5,1) node[above] {1};
  \draw (-0.5,3) node[above] {2};
  \draw (1,1.5) node[] {$n_1$};
  \draw (1,3.5) node[] {$n_2$};
  \draw (0,4.2) -- (0,5);
  \draw (3,4.2) -- (3,5);
  \draw[latex-latex] (0,4.5) -- (3,4.5) node[above, pos=0.5] {$L_2$};
  \draw (0,-0.2) -- (0,-1);
  \draw (4,-0.2) -- (4,-1);
  \draw[latex-latex] (0,-0.7) -- (4,-0.7) node[above, pos=0.5] {$L_1$};
  \end{tikzpicture}
  \captionof{figure}{Problema \ref{p:PO202}\label{f:PO202}}
  \end{center}
\end{minipage}
%
\pma{\label{p:PO201}
En la figura \ref{f:PO201} se muestra el recorrido de dos haces de luz al reflejarse en superficies planas que están sumergidas en agua ($n = 1.3422$). Los haces tienen una longitud de onda en el vacío de $\SI{410}{nm}$ y están inicialmente en fase. \textit{a}) ¿Cuál es el valor mínimo de $L$ que hace que los haces estén en contrafase al salir de esa región? \textit{b}) ¿Y el segundo menor valor?
\rta{.90}{\textit{a}) $\SI{38.18}{nm}$; \textit{b}) $\SI{114.55}{nm}$}}
%
\begin{minipage}[c]{0.5\textwidth}
\begin{center}
  \begin{tikzpicture}[line cap=round,line join=round]
    \coordinate (C1)  at (0,0);
    \coordinate (C2)  at (0,1.5);
    \draw (-2,1.5) node[left] {Haz 2};
    \ray(-2,1.5)(C2)(0.5)(1);
    \ray(C2)(C1)(0.5)(1);
    \ray(C1)(2,0)(1)(1);
    \planemirror[cyan,opacity=0.2](C1)(1)(-45)(0.1);
    \planemirror[cyan,opacity=0.2](C2)(1)(135)(0.1);
    \draw[dashed] (-2,0) -- (-0.5,0);
    \draw[latex-latex] (-1.5,0) -- (-1.5,1.5) node[left, pos=0.5] {$L$};

    \coordinate (C3)  at (1.5,4.5);
    \coordinate (C4)  at (1.5,3);
    \coordinate (C5)  at (0,3);
    \coordinate (C6)  at (0,6);
    \draw (-2,4.5) node[left] {Haz 1};
    \ray(-2,4.5)(C3)(0.4)(1);
    \ray(C3)(C4)(0.5)(1);
    \ray(C5)(C6)(0.75)(1);
    \ray(C4)(C5)(0.5)(-1);
    \ray(C6)(2,6)(1)(1);
    \planemirror[cyan,opacity=0.2](C3)(1)(135)(0.1);
    \planemirror[cyan,opacity=0.2](C4)(1)(45)(0.1);
    \planemirror[cyan,opacity=0.2](C5)(1)(-45)(0.1);
    \planemirror[cyan,opacity=0.2](C6)(1)(225)(0.1);
    \draw[dashed] (-2,6) -- (-0.5,6);
    \draw[dashed] (-2,3) -- (-0.5,3);
    \draw[dashed] (0,6.2) -- (0,7);
    \draw[dashed] (1.5,6.2) -- (1.5,7);
    \draw[latex-latex] (-1.5,3) -- (-1.5,4.5) node[left, pos=0.5] {$L$};
    \draw[latex-latex] (-1.5,4.5) -- (-1.5,6) node[left, pos=0.5] {$L$};
    \draw[latex-latex] (0,6.6) -- (1.5,6.6) node[above, pos=0.5] {$L$};
  \end{tikzpicture}
  \captionof{figure}{Problema \ref{p:PO201}\label{f:PO201}}
\end{center}
\end{minipage}
%
\begin{minipage}[c]{0.5\textwidth}
  \begin{center}
  \begin{tikzpicture}[line cap=round,line join=round]
  \planemirror[cyan,opacity=0.2](0,0)(6)(0)(0.2);
  \fill[gray,opacity=0.3] (2,0) -- (2,3.5) -- (2.1,3.5) -- (2.1,0) -- cycle;
  \draw (2,0) -- (2,3.5) node[above] {Pantalla};
  \coordinate (F)  at (-2,1.75);
  \draw[fill=gray] (F) circle(2.5pt) node[above left] {Fuente};
  \draw[latex-latex] (-2,0) -- (-2,1.7) node[left, pos=0.5] {$h$};
  \draw (0,-0.5) node[] {Espejo};
  \draw[latex-latex] (-2,-1) -- (2,-1) node[below, pos=0.5] {$s$};
  \draw (-2,-0.3) -- (-2,-1.2);
  \draw (2,-0.3) -- (2,-1.2);
  \coordinate (P)  at (2,2.7);
  \filldraw[] (P) circle(2pt) node[above right] {$P$};
  \draw (P) + (0.3,0) --++ (1,0);
  \draw[latex-latex] (P) + (0.7,0) --++ (0.7,-2.7) node[right, pos=0.5] {$y$};
  \ray(F)(P)(0.5)(1);
  \coordinate (x) at (-0.455,0);
  \ray(x)(P)(0.5)(1);
  \ray(F)(x)(0.5)(1);
  \draw (-0.3,2.5) node[] {1};
  \draw (0.8,0.9) node[] {2};
  \end{tikzpicture}
  \captionof{figure}{Problema \ref{p:PO203}\label{f:PO203}}
  \end{center}
\end{minipage}
%
\pma{\label{p:PO203}
Determine el desfasaje entre los rayos 1 y 2 que llegan al punto $P$ (ver figura~\ref{f:PO203}) como función de su altura $y$ respecto del espejo. El sistema está inmerso en un medio de índice $n$.\\
\rta{.87}{$\Delta\phi=\frac{2\pi n}{\lambda_0} \left( \sqrt{(y+h)^2+s^2}-\sqrt{(y-h)^2+s^2} \right) \pm \pi $}}


% \pma{\label{p:PO204}
% Determine el desfasaje entre los rayos 1 y 2 que llegan al punto $P$ en el dispositivo de la figura \ref{f:PO204}. El material que atraviesa el rayo 1 tiene un índice de refracción $n'$ y el sistema está inmerso en un medio de índice de refracción $n$.\\
% \rta{.75}{$\Delta\phi=\frac{2\pi}{\lambda_0}\left(n\sqrt{4h^2+s^2}-n(s-d)-n' d\right) \pm \pi$}}
%
% \begin{center}
%   \begin{tikzpicture}[line cap=round,line join=round]
%   \planemirror[cyan,opacity=0.2](0,0)(6)(0)(0.2);
%   \fill[gray,opacity=0.3] (2,0) -- (2,3.5) -- (2.1,3.5) -- (2.1,0) -- cycle;
%   \draw (2,0) -- (2,3.5) node[above] {Pantalla};
%   \coordinate (F)  at (-2,1.75);
%   \draw[fill=gray] (F) circle(2.5pt) node[above left] {Fuente};
%   \draw[latex-latex] (-2,0) -- (-2,1.7) node[left, pos=0.5] {$h$};
%   \draw (0,-0.5) node[] {Espejo};
%   \draw[latex-latex] (-2,-1) -- (2,-1) node[below, pos=0.5] {$s$};
%   \draw (-2,-0.3) -- (-2,-1.2);
%   \draw (2,-0.3) -- (2,-1.2);
%   \coordinate (P)  at (2,1.75);
%   \filldraw[] (P) circle(2pt) node[above right] {$P$};
%   \draw (P) + (0.3,0) --++ (1,0);
%   \draw[latex-latex] (P) + (0.7,0) --++ (0.7,-1.75) node[right, pos=0.5] {$h$};
%   \fill[gray,opacity=0.3] (-0.5,2.2) -- (0.5,2.2) -- (0.5,1.3) -- (-0.5,1.3) -- cycle;
%   \draw[] (-0.5,2.2) -- (0.5,2.2) -- (0.5,1.3) -- (-0.5,1.3) -- cycle;
%   \draw (0,1.75) node[] {$n'$};
%   \ray(F)(-0.5,1.75)(0.5)(1);
%   \ray(0.5,1.75)(P)(0.5)(1);

%   \coordinate (x) at (0,0);
%   \ray(x)(P)(0.5)(1);
%   \ray(F)(x)(0.5)(1);
%   \draw (1,2) node[] {1};
%   \draw (1,0.5) node[] {2};
%   \draw (-0.5,2.4) -- (-0.5,3);
%   \draw (0.5,2.4) -- (0.5,3);
%   \draw[latex-latex] (-0.5,2.8) -- (0.5,2.8) node[above, pos=0.5] {$d$};
% \end{tikzpicture}
% \captionof{figure}{Problema \ref{p:PO204}\label{f:PO204}}
% \end{center}




  % \section{Experiencia de Young}
\rfigure
%
\pma{\label{p:P208}
Con el objetivo de determinar la longitud de onda de una fuente desconocida se realiza un experimento de interferencia de Young con una separación entre rendijas de $0.5$~mm y la pantalla situada a $1.75$~m. Sobre la pantalla se forman franjas brillantes consecutivas cuyos puntos medios distan $2.1$~mm. ¿Cuál es la longitud de onda de la luz utilizada?\\ \rta{.77}{$\lambda=600$~nm, en el medio en el que se realiza el experimento}}
%
\pma{
Se realiza el experimento de Young con luz de longitud de onda igual a $\SI{502}{\nano\metre}$. Se miden con cuidado las franjas sobre una pantalla que está a $\SI{1.20}{\metre}$ de la doble ranura, y se determina que el centro de la vigésima franja brillante está a $\SI{10.6}{\milli\metre}$ del centro de la franja brillante central. ¿Cuál es la separación entre las dos ranuras?\\
\rta{.95}{$\SI{1.14}{\milli\metre}$}}
%
\pma{
Dos ranuras muy angostas están separadas $\SI{1.80}{\micro\metre}$, colocadas a $\SI{35.0}{\centi\metre}$ de una pantalla y se iluminan con luz coherente de $\lambda = \SI{550}{\nano\metre}$. \textit{a}) ¿Cuál es la distancia entre la segunda y tercera líneas brillantes del patrón de interferencia? \textit{b}) ¿Cuál es la distancia entre la segunda y tercera líneas oscuras del patrón de interferencia? \textit{c}) Calcular nuevamente dichas distancias si el aparato completo (ranuras, pantalla y el espacio intermedio) se sumerge en agua, con un índice de refracción igual a $1.33$ para esta longitud de onda.\\
\rta{.95}{\textit{a}) $\SI{53.27}{\centi\metre}$; \textit{b}) $\SI{23.38}{\centi\metre}$; \textit{c}) $\SI{15.18}{\centi\metre}$ y $\SI{11.71}{\centi\metre}$}}
%
\pma{
Dos ranuras paralelas delgadas que están separadas $\SI{0.0116}{\milli\metre}$ son iluminadas por un rayo láser con longitud de onda de $\SI{585}{\nano\metre}$. \textit{a}) En una pantalla lejana muy grande, ¿cuál es el número total de franjas brillantes? (Sugerencia: Pregúntese cuál es el valor más grande que puede tener $\sin\theta$. ¿Qué le dice esto acerca de cuál es el valor máximo de $m$?). \textit{b}) ¿A qué ángulo con respecto a la dirección original del rayo se presentará la franja más distante de la franja brillante del centro?\\
\rta{.95}{\textit{a}) 39; \textit{b}) $\SI{73.37}{\degree}$}}
%
\pma{\label{p:P209}
En un experimento de doble rendija se observan franjas de interferencia utilizando luz de sodio ($\lambda_0= 589$~nm). ¿Cuál debería ser la longitud de onda utilizada para que la separación angular entre ellas sea 10\% mayor? Suponga válida la aproximación paraxial.\\
\rta{.95}{$\lambda_0 =647.9$~nm}}
%
\pma{\label{p:P213}
Un haz de luz monocromática incide perpendicularmente sobre cuatro rendijas muy estrechas e igualmente espaciadas. Si se cubren las dos rendijas centrales, el máximo de interferencia de cuarto orden se ve bajo el ángulo de 30º. ¿Bajo qué ángulo se ve el máximo de primer orden si se cubren las dos rendijas de los extremos, es decir, dejando las dos centrales abiertas?\\  \rta{.95}{$22.024$º}}
%
\pma{\label{p:P215}
Se hace incidir normalmente luz de longitud de onda $\lambda=632.8$~nm (en el vacío) procedente de un láser de helio-neón sobre un plano que contiene dos rendijas. El primer máximo de interferencia se encuentra a 8~cm del máximo central, cuando se observa el patrón de interferencia en una pantalla de 1~m de ancho (centrada en el máximo de orden 0) situada a 2~m de distancia de las rendijas. \textit{a}) Calcule la separación entre las rendijas. \textit{b}) ¿Cuántos máximos de interferencia se observan en la pantalla? \textit{c}) ¿Qué máximos se observan en la pantalla si la luz incide con un ángulo de 8º sobre las rendijas? \textit{d}) Lo mismo que en \textit{c}), pero con la zona entre las rendijas y la pantalla llena de agua ($n_{agua} = 4/3$).\\ \rta{.95}{\textit{a}) $15.84\,\mu$m ($15.82\,\mu$m usando aproximación paraxial); \textit{b}) 13; \textit{c}) 12 máximos, desde $m=\pm 2$ hasta $m=\mp 9$; \textit{d}) 16 máximos, de $m=\pm 4$ a $m=\mp 11$}}
%
\pma{
Suponga que usted ilumina dos ranuras delgadas con luz coherente monocromática en el aire, y determina que produce su primer mínimo de interferencia en $\SI{35.20}{\degree}$ a ambos lados del punto brillante central. Luego sumerge las ranuras en un líquido transparente, las ilumina con la misma luz y determina que el primer mínimo ahora ocurre en $\SI{19.46}{\degree}$. ¿Cuál es el índice de refracción de este líquido?\\
\rta{.95}{$1.73$}}
%
\pma{\label{p:P212}
A través de dos ranuras angostas separadas por una distancia de $\SI{0.300}{\milli\metre}$ pasa luz coherente que contiene dos longitudes de onda, $\SI{660}{\nano\metre}$ (rojo) y $\SI{470}{\nano\metre}$ (azul), y se observa el patrón de interferencia en una pantalla colocada a $\SI{5.00}{\metre}$ de las ranuras. ¿Cuál es la distancia en la pantalla entre las franjas brillantes de primer orden para las dos longitudes de onda?\\
\rta{.95}{$\SI{3.17}{\milli\metre}$}}
%


  % \section{Difracción}
\rfigure
% \setcounter{equation}{0}
%
\pma{\label{p:P231}
Considere un haz de ondas planas que incide sobre una placa que contiene una rendija. Si la intensidad del máximo central en el patrón de difracción es $I_\text{max}$, determine la intensidad (relativa a la máxima) de los primeros 3 máximos secundarios de difracción.\\
\rta{.95}{$I_1=0.047\,I_\text{max}$, $I_2=0.016\,I_\text{max}$, $I_3=0.008\,I_\text{max}$}
}
%
\pma{\label{p:P237}
Los rayos paralelos de un haz de luz de longitud de onda $\lambda_0=650$~nm pasan por una pantalla en la cual se halla una rendija de $0.5$~mm de ancho y 3~cm de alto. El patrón de difracción generado por esta rendija se observa en una pantalla ubicada 50~cm a continuación de la rendija. Calcular la distancia, medida sobre la pantalla, entre el máximo principal del patrón de difracción y \textit{a}) los primeros mínimos, \textit{b}) los primeros máximos secundarios.\\
\rta{.95}{\textit{a}) $\pm0.65$~mm; \textit{b}) $\pm0.93$~mm}}
%
\pma{\label{p:P238}
Se observa el patrón de difracción generado por una rendija de $0.25$~mm de ancho sobre una pantalla situada a una distancia de $2.5$~m de ella. La campana principal de difracción tiene un ancho de 11~mm. ¿Cuál es la longitud de onda de la luz utilizada?\\
\rta{.95}{550~nm}
}
%
\pma{\label{p:P239}
Se ilumina con luz infrarroja proveniente de un láser de He-Ne ($\lambda_0= 1152.2$~nm en el vacío) una pantalla que contiene una rendija estrecha, y se determina que el centro de la décima banda oscura en el patrón de difracción de Fraunhofer se encuentra en un ángulo de $\SI{6.3}{\degree}$ respecto del eje central. \textit{a}) Calcule el ancho de la rendija. \textit{b}) ¿Para qué ángulo aparecerá la décima franja oscura si todo el arreglo se sumerge en agua $\left(n_a(\lambda_0) = 1,326\right)$ en lugar de aire?\\ \rta{.95}{\textit{a}) $a=\SI{105}{\micro\meter}$; \textit{b}) $\theta_{10}=\SI{4.747}{\degree}$}}
%
\pma{\label{p:P240}
Calcule el ancho $a$ de la ranura rectangular que producirá en su patrón de difracción de campo lejano una campana principal con un ancho angular $\delta\theta$ de \textit{i}) $\SI{30}{\degree}$, \textit{ii}) $\SI{45}{\degree}$, \textit{iii}) $\SI{90}{\degree}$, \textit{iv}) $\SI{180}{\degree}$, al ser iluminada con de longitud de onda 550~nm.\\
\rta{.95}{\textit{i}) $\SI{2.125}{\micro\meter}$; \textit{ii}) $\SI{1.437}{\micro\meter}$; \textit{iii}) $\SI{0.778}{\micro\meter}$; \textit{iv}) $\SI{0.550}{\micro\meter}$}}
%
\pma{\label{p:P241}
El patrón de difracción producido por una única rendija, cuando se la ilumina con luz de 550~nm, es observado sobre una pantalla ubicada a 40~cm a continuación de la placa que contiene a la rendija. Si la distancia entre el primer y quinto mínimo es de $0.4$~mm, determinar el ancho de la rendija.
\rta{.95}{\textit{a}) $a=2.2$~mm}}
%
\begin{minipage}[t]{.5\textwidth}
  \textbf{Pantallas complementarias}. Consideremos dos superficies $\Sigma$ y $\Sigma'$ de manera tal que en los puntos donde la superficie $\Sigma$ es transparente, $\Sigma'$ es opaca y viceversa, como se muestra en la figura \ref{f:babinet}. Por ejemplo, si $\Sigma$ es un agujero circular en una pantalla opaca, entonces $\Sigma'$ es un disco opaco del mismo tamaño y posición en una pantalla transparente. Este tipo de pantallas se denominan complementarias. Se puede mostrar que el patrón de difracción de la pantalla $\Sigma$ es el mismo que produce la pantalla $\Sigma'$ (salvo en la dirección de la onda incidente). Este hecho se conoce con el nombre de \textit{principio} o \textit{teorema de Babinet}.
\end{minipage}
\hfill
%\begin{figure}
\begin{minipage}[t]{.55\textwidth}
\strut\vspace*{-\baselineskip}

  \pgfdeclarelayer{layer1}
  \pgfdeclarelayer{layer2}
  \pgfdeclarelayer{layer3}
  \pgfsetlayers{main, layer3, layer2, layer1}

  \begin{center}
  \begin{tikzpicture}[x={(1cm,0.4cm)}, y={(8mm, -3mm)}, z={(0cm,1cm)}, line cap=round, line join=round, scale=0.75]
    \begin{pgfonlayer}{layer1}
      \object{3}
      \ray(0,-1,0)(2.5,-1,0)(0.5)(1);
      \ray(0,1,0)(2.5,1,0)(0.5)(1);
    \end{pgfonlayer}
    \begin{pgfonlayer}{layer3}
      \image{8}
    \end{pgfonlayer}
    \begin{pgfonlayer}{layer2}
      \lens{5.4}
    \end{pgfonlayer}

    \draw (3,0,-2) node[] {Objeto 1};
    \draw (5.4,0,-1.5) node[] {Lente};
    \draw (8,0,-2) node[] {Pantalla 1};

    \begin{pgfonlayer}{layer1}
      \begin{scope}[shift={(0,0,-4.5)}]
        \objecttriangle{3};
        \ray(0,-1,0)(2.5,-1,0)(0.5)(1);
        \ray(0,1,0)(2.5,1,0)(0.5)(1);
      \end{scope}
    \end{pgfonlayer}
    \begin{pgfonlayer}{layer3}
      \begin{scope}[shift={(0,0,-4.5)}]
        \image{8}
        \lens{5.4}
      \end{scope}
    \end{pgfonlayer}

    \begin{scope}[shift={(0,0,-4.5)}]
      \draw (3,0,-2) node[] {Objeto 2};
      \draw (5.4,0,-1.5) node[] {Lente};
      \draw (8,0,-2) node[] {Pantalla 2};
    \end{scope}
  \end{tikzpicture}
  \captionof{figure}{Pantallas complementarias.}\label{f:babinet}
  \end{center}
\end{minipage}
%
\pma{\label{p:P234}
La luz de un láser de He-Ne ($\lambda_0=632.8$~nm) se dirige hacia un cabello humano, en un experimento para medir su diámetro examinando el patrón de difracción. El cabello está montado sobre un bastidor y el patrón de difracción se observa sobre una pantalla que se encuentra a $0.75$~m del mismo. Si el ancho de la campana principal es de $1.46$~cm, ¿cuál es el diámetro del cabello?\\
\rta{.95}{$\SI{65}{\micro\meter}$}}
%
\pma{\label{p:P242}
¿Cuántos máximos de interferencia estarán contenidos dentro de la campana principal de difracción en el diagrama de interferencia-difracción de \textbf{dos rendijas}, si la separación $d$ entre ellas es 5 veces su ancho $a$? ¿Y si es $6.5$ veces?
\rta{.95}{9; 13}}
%
\pma{\label{p:P243}
Se observa un diagrama de interferencia-difracción de Fraunhofer producido al iluminar con luz de longitud de onda $\lambda = 500$~nm dos rendijas de ancho $a$ separadas por $0.1$~mm. \textit{a}) Calcule el ancho $a$ de las rendijas si el décimo máximo de interferencia se produce en el mismo ángulo que el segundo mínimo de difracción. \textit{b}) En ese caso, ¿cuántas franjas brillantes se verán en la primera campana secundaria de difracción? ¿A qué órdenes corresponden?\\
\rta{.95}{\textit{a}) $a=0.02$~mm; \textit{b}) 4 ($m=6, 7, 8$ y $9$)}}
%
\pma{\label{p:P244}
Un haz de luz de 550~nm de longitud de onda ilumina dos rendijas de ancho $0.025$~mm y separación $0.15$~mm. \textit{a}) ¿Cuántos máximos de interferencia entran dentro de la campana principal de difracción? \textit{b}) ¿Cuál es el cociente entre la intensidad del tercer máximo de interferencia a un lado de la línea central y la intensidad de este máximo central?\\
\rta{.95}{\textit{a}) El central más 5 máximos a cada lado, un total de 11; \textit{b}) $\frac{4}{\pi^2}=0.4053$}}
%
\pma{\label{p:P245}
La luz de proveniente de una fuente de longitud de onda $\lambda$=475 nm pasa a través de una doble rendija, produciendo un patrón de interferencia-difracción cuyo gráfico de intensidad $I(\theta)$ se puede ver en la figura \ref{f:P245}. \textit{a}) Calcule el ancho de las ranuras y la separación entre ellas. \textit{b}) Verifique que las intensidades de los máximos de interferencia para $m=1$ y $m=2$ son las correctas. \textit{c}) ¿Cuál es el valor de $I_0$?\\
\rta{.95}{\textit{a}) $a=5.45\,\mu$m y $d=21,8\,\mu$m; \textit{b}) los valores calculados para el primer y segundo máximo: $I_1=5.674$~mW/cm\textsuperscript{2} y $I_2=2.837$~mW/cm\textsuperscript{2} respectivamente, y los leídos del gráfico: $5.67\pm0.07$~mW/cm\textsuperscript{2} y $2.89\pm0.07$~mW/cm\textsuperscript{2}, no muestran diferencias significativas entre sí; \textit{c}) $I=1.75$~mW/cm\textsuperscript{2}}}
%
\begin{center}
  \begin{tikzpicture}[scale=1]
    \begin{axis}[
                 every major x tick/.append style={thick,blue},
                 clip=false,
                 grid=both,
                 minor x tick num=1,        %un minor tick es decir 0.5
                 minor y tick num=1,
                 xmin=0, xmax=15,           %min y max para los ejes, NO PARA EL DOMINIO
                 ymin=0, ymax=8,
                 %axis y line=center,        %alinea el eje al centro de la figura
                 %axis x line=middle,        %sino pone 2 ejes x
                 xtick  align=center,
                 xlabel={Ángulo~[grados]},
                 ylabel={Intensidad~[$\si{\milli\watt/\metre^2}$]},
                 width=12cm,
                 height=8cm
                ];
    \addplot [color=green!70!black, very thick] [samples= 400, domain=0.001:15]  {7*(cos(180*sin(x)*4/sin(5))*sin(180*sin(x)*1/sin(5))/(pi*sin(x)*1/sin(5)))^2};
    \end{axis}
  \end{tikzpicture}
  \captionof{figure}{Problema \ref{p:P245}\label{f:P245}}
\end{center}
%
\pma{\label{p:P267}
Un haz de luz de longitud de onda (en el vacío) $\lambda_0= 500$~nm incide sobre una placa con dos rendijas, separadas entre sí una distancia $d = 25 \mu$m y de ancho $a = \frac15 d$. La placa se encuentra en la división entre dos medios de índice de refracción $n_1 = 1.20$ y $n_2 = 1.28$, respectivamente. Al hacer incidir el haz con un ángulo $\theta_i$, se observa que el patrón de intensidades se desplaza 30 franjas respecto de la situación con incidencia normal. La pantalla de observación está ubicada a $1$~m de distancia de la placa y está centrada respecto del eje del sistema. \textit{a}) Calcule el ángulo de incidencia $\theta_i$. \textit{b}) ¿A qué distancia del eje del sistema se encuentra el centro de la campana principal de difracción? \textit{c}) Calcule la irradiancia en el centro de la pantalla.\\
\rta{.95}{\textit{a}) $\theta_i=30$º; \textit{b}) $53$~cm; \textit{c}) $I=0$}}
%
\pma{\label{p:P305}
Un haz de luz de longitud de onda (en el vacío) $\lambda_0= 600$~nm incide con un ángulo de $0.967$º sobre una placa que contiene dos rendijas de $20\,\mu$m de ancho y separadas $80\,\mu$m.  Se observa el patrón de interferencia en una pantalla situada a 1 m de la placa con las rendijas. \textit{a}) ¿Cuánto mide la interfranja?
\textit{b}) Calcule la intensidad que se registra en el centro de la pantalla, sabiendo que desde cada rendija sale luz con $5$~W/m\textsuperscript{2} de intensidad.\\
\rta{.95}{\textit{a}) $7.5$~mm; \textit{b}) $3.08$~W/m\textsuperscript{2}}}
  % \section{Polarización}
\rfigure
%
\pma{\label{p:P174}
Una onda luminosa que se propaga a lo largo del eje $z$ incide sobre un polarizador lineal, contenido siempre en el plano $xy$. Si la intensidad de la luz que llega al polarizador es $I_0$, discuta cómo varía (o no) la intensidad luminosa transmitida al girar el eje de transmisión, en los casos en que la luz incidente sea:
\bemca
 \item circularmente polarizada dextrógira; %a
 \item circularmente polarizada levógira;  %b
 \item elípticamente polarizada dextrógira; %c
 \item elípticamente polarizada levógira; %d
 \item linealmente polarizada; %e
 \item luz natural. %f
\eemca
\noindent
\rta{.95}{\textit{a}),\textit{ b}),\textit{ f}) La intensidad no varía y vale $\frac12 I_0$; \textit{c}),\textit{ d}) la intensidad varía entre $I_{min}\neq 0$ e $I_{max}$ (con $I_{max} + I_{min} = I_0$); \textit{e}) la intensidad varía entre 0 e $I_0$}
}
%
\pma{\label{p:P175}
Un haz de luz no polarizada de irradiancia $I_0$ pasa a través de una secuencia de dos polarizadores lineales ideales. ¿Cuál debe ser la orientación relativa entre sus ejes de transmisión, si el haz emergente tiene una irradiancia de \textit{a}) $I_0/2$; \textit{b}) $I_0/4$?\\
\rta{.95}{Sus ejes de transmisión deben estar \textit{a}) paralelos; \textit{b}) a 45º}}
%
\pma{\label{p:PO176}
Dos láminas polarizadoras tienen sus ejes de transmisión cruzados perpendicularmente con $\theta_1=0$ y $\theta_3=\frac{\pi}{2}$ (ver figura \ref{f:PO176}). Entre ellas se inserta una tercera lámina de modo que su eje de transmisión forme un ángulo $\theta_2$ con el de la primera lámina. Si sobre la primer lámina incide luz natural, demuestre que la intensidad transmitida a través de las tres láminas es máxima cuando $\theta_2=45$º.}
%
\pma{\label{p:PO177}
La lámina polarizadora intermedia del problema \ref{p:PO176} está girando alrededor del eje $z$ con velocidad angular $\omega=4\pi$~rad/s. Si sobre la primera incide luz natural de intensidad 8~W/m$^2$, determine los valores máximo y mínimo de la intensidad luminosa transmitida a través de las tres láminas y cada cuánto tiempo se observarán los máximos de intensidad. \\
\rta{.95}{$I_{max}=1$~W/m$^2$, $I_{min}=0$~W/m$^2$; los máximos en la irradiancia se observan cada $0.125$~s}}
%
\pma{\label{p:PO178}
Un haz de luz no polarizada se envía a través del sistema de tres láminas polarizadoras del problema \ref{p:PO176}, cuyos ejes de transmisión forman ángulos $\theta_1=40$º, $\theta_2=20$º, $\theta_3=40$º. ¿Qué porcentaje de la luz que llega al primer polarizador sale por el tercero?\\
\rta{.75}{$3.125$~\%}}

\pgfdeclarelayer{layer1}
\pgfdeclarelayer{layer2}
\pgfdeclarelayer{layer3}
\pgfsetlayers{main, layer3, layer2, layer1}

\begin{center}
\begin{tikzpicture}[x={(1cm,0.4cm)}, y={(8mm, -3mm)}, z={(0cm,1cm)}, line cap=round, line join=round, scale=0.75]
  % \begin{pgfonlayer}{layer1}
  %   \object{3}
  %   \ray(0,-1,0)(2.5,-1,0)(0.5)(1);
  %   \ray(0,1,0)(2.5,1,0)(0.5)(1);
  % \end{pgfonlayer}
  % \begin{pgfonlayer}{layer3}
  %   \image{8}
  % \end{pgfonlayer}
  \begin{pgfonlayer}{layer1}
    \lens{2}
    \begin{scope}[canvas is yz plane at x=2]
      \draw[dashed] (0.7,-0.8) -- (-1.4,1.6);
      \draw (0,1.2) -- (0,1.8);
      \draw[-latex] (0,1.4) arc (90:133:1.4) node[above, pos=0.5] {$\theta_3$};
    \end{scope}
    \ray(0,0,0)(2,0,0)(0.5)(-1);

  \end{pgfonlayer}

  \begin{pgfonlayer}{layer2}
    \lens{5}
    \begin{scope}[canvas is yz plane at x=5]
      \draw[dashed] (0.7,0.8) -- (-1.4,-1.6);
      \draw (0,-1.2) -- (0,-1.8);
      \draw[-latex] (0,-1.4) arc (270:227:1.4) node[below, pos=0.5] {$\theta_2$};
    \end{scope}
    \ray(5,0,0)(2,0,0)(0.5)(-1);
  \end{pgfonlayer}

  \begin{pgfonlayer}{layer3}
    \ray(11,0,0)(8,0,0)(0.5)(-1);
    \lens{8}
    \begin{scope}[canvas is yz plane at x=8]
      \draw[dashed] (0.7,-0.8) -- (-1.4,1.6);
      \draw (0,1.2) -- (0,1.8);
      \draw[-latex] (0,1.4) arc (90:133:1.4) node[above, pos=0.5] {$\theta_1$};
    \end{scope}
    \ray(8,0,0)(5,0,0)(0.5)(-1);
  \end{pgfonlayer}

  \draw[-latex] (10,0,0) -- (10,0,1) node[right] {$y$};
  \draw[-latex] (10,0,0) -- (10,1,0) node[right] {$x$};
  
\end{tikzpicture}
\captionof{figure}{Problemas \ref{p:PO176}, \ref{p:PO177}, \ref{p:PO178}.}\label{f:PO176}
\end{center}
%
\pma{\label{p:P179}
Se desea girar 90º el plano de polarización de un haz de luz polarizado linealmente de intensidad $I_0$. ¿Cómo podría hacerse usando únicamente polarizadores lineales? Diseñe un experimento donde la pérdida de la intensidad total sea menor al 40\%. Justifique analíticamente su respuesta.\\
\rta{.95}{Usando $N$ polarizadores, de manera tal que cada uno tenga su eje de transmisión rotado $\pi/2N$ respecto del anterior. A la salida se obtiene un haz de intensidad $I=I_0\cos^{2N}\left(\frac{\pi}{2N}\right)$ y una rotación total de $\pi/2$ respecto del haz inicial. Se verifica que $I/I_0>0.6$ para $N\geq5$}}

  % \section{Preguntas sobre óptica para el análisis}
\rfigure
\textit{En esta sección se requiere que se brinden respuestas argumentadas.}

\pma{
Se realiza un experimento de interferencia con dos ranuras, y las franjas se proyectan en una pantalla. Después, todo el aparato se sumerge agua, ¿cómo cambia el patrón de las franjas?
}
%
\pma{
A través de dos ranuras delgadas pasa luz monocromática coherente que se ve en una pantalla alejada. ¿Las franjas brillantes en la pantalla se encontrarán igualmente separadas? Si es así, ¿por qué? Si no, ¿cuáles están más cerca de estar igualmente separadas?
}
%
\pma{
En un patrón de interferencia de dos ranuras sobre una pantalla distante, ¿las franjas brillantes están a la mitad de la distancia que hay entre las franjas oscuras?
}
%
\pma{
Las luces de un automóvil distante, ¿formarían un patrón de interferencia de dos fuentes?
}
%
\pma{
Se iluminan con luz coherente de longitud de onda $\lambda$ dos ranuras estrechas separadas por una distancia $d$. Si $d$ es menor que cierto valor mínimo, no se observan franjas oscuras. Explique lo que sucede. En términos de $\lambda$, indique cuál es este valor mínimo de $d$.
}
%
\pma{
¿Por qué podemos observar fácilmente los efectos de la difracción en el caso de las ondas sonoras y de las ondas en el agua, pero no en el caso de la luz?
}
%
\pma{
A través de una sola ranura de ancho $a$ pasa luz de longitud de onda $\lambda$ y frecuencia $f$. Se observa el patrón de difracción en una pantalla a una distancia $S$ de la ranura. De las acciones siguientes, ¿cuáles reducen la anchura del máximo central? \textit{a}) Disminuir el ancho $a$ de la ranura. \textit{b}) Disminuir la frecuencia $f$ de la luz. \textit{c}) Disminuir la longitud de onda $\lambda$ de la luz. \textit{d}) Disminuir la distancia $S$ de la ranura a la pantalla.
}
%
\pma{
En un experimento de difracción que utiliza ondas con longitud de onda $\lambda$, no habrá mínimos de intensidad (es decir, no habrá franjas oscuras) si la anchura de la rendija es lo suficientemente pequeña. ¿Cuál es el ancho máximo de rendija con el cual ocurre esto?
}
%
% \pma{
% Cuando la luz no polarizada incide en dos polarizadores cruzados, no se transmite luz. Un estudiante afirmó que si se insertaba un tercer polarizador entre los otros dos, habría algo de transmisión. ¿Tiene sentido esto? ¿Cómo podría un tercer filtro incrementar la transmisión?
% }
%


  
\begin{titlepage}
\begin{figure}[ht]
\begin{center}
\vspace{1.5cm}
% Aquí se inserta el escudo o emblema:
\begin{tikzpicture}[scale=1, every node/.style={scale=1}]
    \definecolor{topcolor}{RGB}{255,128,0};
    \definecolor{botcolor}{RGB}{255,128,0};

    \fill[color=topcolor] (1.8273,1.065) arc (30.235:149.765:2.115) -- cycle;
    \fill[color=botcolor] (1.8273,-1.065) arc (-30.235:-149.765:2.115) -- cycle;

    \draw[color=botcolor, thick] (-1.755,-0.625) -- (1.755, -0.625);
    \fill[color=botcolor] (-1.755,-0.0929) -- (1.755, -0.0929) -- (1.755, -0.5405) -- (-1.755,-0.5405) -- cycle;

    % \fill [black](0,-0.7) circle(2pt);

    % \draw (-1.7539,0.93) rectangle (-0.9592,0);
    \fill[] (-1.7539,0.93) arc (180:360:0.39735) -- (-1.1178,0.93) arc (360:180:0.23875) -- cycle;
    \fill[] (-1.7539,0) arc (180:0:0.39735) -- (-1.1178,0) arc (0:180:0.23875) -- cycle;
    \fill (-1.7539,0.3866) rectangle (-0.9592,0.5434);
    \fill (-1.43495,0.93) rectangle (-1.27815,0);

    \fill[color=topcolor] (-0.7795,0.3275) arc[start angle=180, end angle=360,x radius=0.40585, y radius=0.3275] -- (0.0322,0.8506) arc[start angle=0, end angle=180,x radius=0.08925, y radius=0.0794] -- (-0.1463,0.3275) arc[start angle=0, end angle=-180,x radius=0.22735, y radius=0.1604] -- (-0.603, 0.8506) arc[start angle=0, end angle=180,x radius=0.08925, y radius=0.0794] -- cycle;

    \fill[color=topcolor] (0.4003,0.0794) arc[start angle=180, end angle=360,x radius=0.08925, y radius=0.0794] -- (0.5788,0.7515) -- (0.7986,0.7515) arc[start angle=-90, end angle=90,x radius=0.0794, y radius=0.08925] -- (0.1805,0.93) arc[start angle=90, end angle=270,x radius=0.0794, y radius=0.08925] -- (0.4003, 0.7515) -- cycle;

    \fill[color=topcolor] (0.9455,0.0794) arc[start angle=180, end angle=360,x radius=0.08925, y radius=0.0794] -- (1.124,0.608) -- (1.57,0.0267) arc[start angle=-153.217, end angle=0,x radius=0.09411, y radius=0.07059] -- (1.755,0.8506) arc[start angle=0, end angle=180,x radius=0.08925, y radius=0.0794] -- (1.5765,0.3191) -- (1.124,0.9004) arc[start angle=26.783, end angle=180,x radius=0.09411, y radius=0.07059] -- cycle;

    \node [scale=0.8,color=white] at (-1.3822,-0.31) {{\fontfamily{cmss}\selectfont \textbf{A}}};
    \node [scale=0.8,color=white] at (-1.0751,-0.31) {{\fontfamily{cmss}\selectfont \textbf{V}}};
    \node [scale=0.8,color=white] at (-0.768,-0.31) {{\fontfamily{cmss}\selectfont \textbf{E}}};
    \node [scale=0.8,color=white] at (-0.4609,-0.31) {{\fontfamily{cmss}\selectfont \textbf{L}}};
    \node [scale=0.8,color=white] at (-0.1536,-0.31) {{\fontfamily{cmss}\selectfont \textbf{L}}};
    \node [scale=0.8,color=white] at (0.1536,-0.31) {{\fontfamily{cmss}\selectfont \textbf{A}}};
    \node [scale=0.8,color=white] at (0.4605,-0.31) {{\fontfamily{cmss}\selectfont \textbf{N}}};
    \node [scale=0.8,color=white] at (0.7676,-0.31) {{\fontfamily{cmss}\selectfont \textbf{E}}};
    \node [scale=0.8,color=white] at (1.0749,-0.31) {{\fontfamily{cmss}\selectfont \textbf{D}}};
    \node [scale=0.8,color=white] at (1.3822,-0.31) {{\fontfamily{cmss}\selectfont \textbf{A}}};

    \node [scale=1] at (0,-0.845) {{\fontfamily{cmss}\selectfont UDB Física}};

\end{tikzpicture}

% \input{termodinamica/termo_logoFRA}

\vspace{-1cm}
\end{center}
\end{figure}

\begin{center}
{\LARGE UNIVERSIDAD TECNOLÓGICA NACIONAL}\\
\vspace{0.25cm}
{\LARGE Facultad Regional Avellaneda}\\
\vspace{1cm}
{\Huge Física II - \comision}
% {\Huge Física II}

\vspace{1.5cm} {\LARGE Guía de problemas de la unidad II}\\
% \vspace{1.5cm} {\LARGE Guía de problemas}\\
\vspace{1cm}
{\Huge \bf \color[RGB]{255,128,0} Óptica}
% {\Huge \bf \color[RGB]{0,121,138} Óptica}


\vspace*{\fill}
\vspace{0.75cm} {Año \anio}
\vspace{0.75cm}

% \begin{figure}[h]
%   \centering
%   \includegraphics[scale=0.28]{portada1-03.jpg}
%   \hfill
%   \includegraphics[scale=0.28]{portada1-01.jpg}
%     \vspace{3mm}  Luz atravesando un medio material (gotas de agua).
% \end{figure}

\end{center}


\end{titlepage}

\newpage
\pagenumbering{roman}
\tableofcontents

\newpage
\pagenumbering{arabic}

  \begin{center}
   {\scshape \Huge  Óptica\par}
   \vspace{0.5 cm}
\end{center}

% \ifthenelse{\equal{\unidaddos}{false}}{
% \part{\scshape Óptica}
% \vspace{0.5 cm}
% }

\section{Ondas}
\rfigure
%
\pma{\label{p:P170}
\textit{a}) Ondas de radio de onda media y de frecuencia modulada tienen frecuencias del orden de los $\SI{1200}{kHz}$ y de 120 kHz, respectivamente. Hallar sus
correspondientes longitudes de onda en el vacío. \textit{b}) El oído humano es capaz de percibir sonidos de frecuencias comprendidas entre $\SI{20}{Hz}$ y $\SI{20000}{Hz}$. Determinar las longitudes de onda correspondientes a estas frecuencias, suponiendo que la velocidad del sonido es $\SI{338}{m/s}$. \textit{c}) La radiación de los hornos a microondas o la de las señales Wi-Fi tienen una frecuencia cercana a los $\SI{2.45}{GHz}$. Calcular la longitud de onda de estas radiaciones.\\
\rta{.95}{\textit{a}) Entre $\SI{250}{\metre}$ y $\SI{2500}{\metre}$; \textit{a}) Entre $\SI{0.0169}{\metre}$ y $\SI{16.9}{\metre}$; \textit{c}) $\SI{0.122}{m}$}
}
%
\pma{
Un hombre que se sienta en el borde de un muelle a pescar y cuenta las ondas de agua que golpean un poste de soporte del muelle, en un minuto cuenta 76 ondas. Si una
cresta en particular viaja $\SI{18}{m}$ en $\SI{6}{s}$, ¿cuál es la longitud de onda de estas ondas?\\
\rta{.95}{$\SI{2.37}{m}$}
}
%
\pma{
Una onda armónica en un hilo tiene una amplitud de $\SI{17}{mm}$, una longitud de onda de $\SI{1.96}{m}$ y una velocidad de $\SI{4.1}{m/s}$. Determinar el período, la frecuencia, la frecuencia angular y el número de onda.\\
\rta{.95}{$f = \SI{2.09}{\hertz}$; $T = \SI{0.48}{\second}$; $\omega = \SI{13.13}{\radian/\second}$; $k = \SI{3.2}{\metre^{-1}}$}
}
%
\pma{
La velocidad de las ondas electromagnéticas en el vacío es $\SI{3E8}{m/s}$. Las longitudes de onda de las ondas electromagnéticas visibles se extienden aproximadamente desde $\SI{400}{nm}$ (luz violeta) hasta $\SI{750}{nm}$ (luz roja). Determinar el rango de frecuencias de luz visible.\\
\rta{.95}{Entre $\SI{4.0E14}{\hertz}$ y $\SI{7.5E14}{\hertz}$}
}
%
\pma{
La intensidad promedio de la luz solar sobre la superficie terrestre es de unos $\SI{700}{W/m^2}$. \textit{a}) Calcular la cantidad de energía que incide sobre un panel solar de $\SI{0.5}{m^2}$ de área en 4 horas. \textit{b}) ¿Qué intensidad tendrá la luz solar, si es concentrada por una lupa sobre una superficie 200 veces menor que la propia de la lupa?\\
\rta{.95}{\textit{a}) $\SI{5.04E6}{J}$; \textit{b}) $\SI{1.4E5}{W/m^2}$}
}


  \section{Superposición coherente entre dos ondas de luz. Interferencia}
\rfigure
%
\pma{
Dos haces de luz linealmente polarizados, con sus campos eléctricos paralelos, se propagan en la misma dirección y sentido. Uno de ellos tiene una intensidad de $\SI{7}{\watt/\metre^2}$ y una frecuencia de 600~THz ($\SI{6E14}{\hertz}$), y el otro una intensidad de $\SI{3}{\watt/\metre^2}$ y una longitud de onda de 500~nm ($\SI{500E-9}{m}$). \textit{a}) Calcule la intensidad resultante si la fase relativa entre ambas ondas se mantiene constante en el tiempo y vale $\varphi=\frac{\pi}{3}$. \textit{b}) Repita el cálculo, si la longitud de onda del segundo haz es 520~nm.\\
\rta{.95}{\textit{a}) Las ondas son coherentes y por lo tanto hay interferencia: $\SI{14.6}{\watt/\metre^2}$; \textit{b}) las ondas son incoherentes, es válida la suma de intensidades: 10 W/m$^2$}}

\pma{
Calcule la mínima diferencia que hay entre los caminos recorridos por dos haces de $600$~nm de longitud de onda en el vacío que al superponerse están en contrafase si: \textit{a}) se propagan en el vacío, \textit{b}) se propagan en agua ($n_{agua}=\frac43$). \textit{Nota}: la luz en ningún caso cambia de medio por el que se propaga ni se refleja y los haces están inicialmente en fase.
\rta{.90}{\textit{a}) $\SI{300}{\nano\metre}$; \textit{b}) $\SI{225}{\nano\metre}$}}
%
\pma{\label{p:PO205}
La figura \ref{f:PO205} muestra dos haces de luz de longitud de onda igual a $\SI{450}{nm}$ que se propagan en aire. En cierta parte del recorrido, el haz 1 atraviesa un objeto traslúcido de longitud $L = \SI{21.4}{\micro\metre}$, con un índice de refracción $n' = 1.45$. \textit{a}) ¿Cuál es la diferencia de fase entre ambos haces luego de que el haz 1 vuelve a propagarse en aire, si inicialmente ambos haces tenían la misma fase? \textit{b}) Encuentre al menos dos valores posibles de $L$ para que los haces estén en contrafase luego de que el haz 1 atraviese el material de índice $n'$.\\
\rta{.90}{\textit{a}) $\SI{134.46}{\radian} = (42\pi+0.4\pi)~\si{\radian}$; \textit{b}) Los distintos valores cumplen $L = (2m-1)\cdot\SI{500}{nm}$ con $m \in \mathbb{N}$}}
%
\pma{\label{p:PO202}
La figura \ref{f:PO202} muestra dos haces de luz de $\SI{650}{nm}$ de longitud de onda que se propagan en aire y cruzan dos capas delgadas de materiales plásticos. El haz 1 atraviesa una distancia $L_1 = \SI{4}{\micro\metre}$ dentro de policarbonato (índice de refracción $n_1 = 1.58$) y el 2 recorre una distancia $L_2 = \SI{3.5}{\micro\metre}$ en acrílico (índice de refracción $n_2 = 1.49$). Considere que ambos haces recorren la misma distancia y que tienen la misma fase antes de ingresar en los plásticos. \textit{a}) ¿Cuál es la diferencia entre los caminos ópticos de cada haz (en unidades de longitud y en múltiplos de la longitud de onda) desde que entran a los plásticos hasta el punto A? \textit{b}) ¿Cuál es la diferencia de fase entre los haces?\\
\rta{.95}{\textit{a}) $\SI{605}{\nano\metre}$ y $0.931\,\lambda_0$; \textit{b}) $\Delta\phi = \SI{5.85}{\radian}$}}
%
\begin{minipage}[c]{0.45\textwidth}
\begin{center}
  \begin{tikzpicture}[line cap=round,line join=round]
  \fill[cyan,opacity=0.3] (0,0) -- (0,2) -- (4,2) -- (4,0) -- cycle;
  \draw (0,0) -- (0,2) -- (4,2) -- (4,0) -- cycle;
  \ray(-1,3)(5,3)(0.5)(1);
  \ray(-1,1)(5,1)(0.5)(1);
  \draw (-0.5,1) node[above] {1};
  \draw (-0.5,3) node[above] {2};
  \draw (1,1.5) node[] {$n'$};
  \draw (0,-0.2) -- (0,-1);
  \draw (4,-0.2) -- (4,-1);
  \draw[latex-latex] (0,-0.7) -- (4,-0.7) node[above, pos=0.5] {$L$};
  \end{tikzpicture}
  \captionof{figure}{Problema \ref{p:PO205}\label{f:PO205}}
\end{center}
\end{minipage}
%
\begin{minipage}[c]{0.45\textwidth}
  \begin{center}
  \begin{tikzpicture}[line cap=round,line join=round]
  \fill[cyan,opacity=0.3] (0,0) -- (0,2) -- (4,2) -- (4,0) -- cycle;
  \fill[orange,opacity=0.3] (0,2) -- (0,4) -- (3,4) -- (3,2) -- cycle;
  \draw (0,0) -- (0,4);
  \draw (0,2) -- (4,2);
  \draw (4,0) -- (4,4) node[above] {$A$};
  \draw (3,2) -- (3,4);
  \ray(-1,3)(5,3)(0.5)(1);
  \ray(-1,1)(5,1)(0.5)(1);
  \draw (-0.5,1) node[above] {1};
  \draw (-0.5,3) node[above] {2};
  \draw (1,1.5) node[] {$n_1$};
  \draw (1,3.5) node[] {$n_2$};
  \draw (0,4.2) -- (0,5);
  \draw (3,4.2) -- (3,5);
  \draw[latex-latex] (0,4.5) -- (3,4.5) node[above, pos=0.5] {$L_2$};
  \draw (0,-0.2) -- (0,-1);
  \draw (4,-0.2) -- (4,-1);
  \draw[latex-latex] (0,-0.7) -- (4,-0.7) node[above, pos=0.5] {$L_1$};
  \end{tikzpicture}
  \captionof{figure}{Problema \ref{p:PO202}\label{f:PO202}}
  \end{center}
\end{minipage}
%
\pma{\label{p:PO201}
En la figura \ref{f:PO201} se muestra el recorrido de dos haces de luz al reflejarse en superficies planas que están sumergidas en agua ($n = 1.3422$). Los haces tienen una longitud de onda en el vacío de $\SI{410}{nm}$ y están inicialmente en fase. \textit{a}) ¿Cuál es el valor mínimo de $L$ que hace que los haces estén en contrafase al salir de esa región? \textit{b}) ¿Y el segundo menor valor?
\rta{.90}{\textit{a}) $\SI{38.18}{nm}$; \textit{b}) $\SI{114.55}{nm}$}}
%
\begin{minipage}[c]{0.5\textwidth}
\begin{center}
  \begin{tikzpicture}[line cap=round,line join=round]
    \coordinate (C1)  at (0,0);
    \coordinate (C2)  at (0,1.5);
    \draw (-2,1.5) node[left] {Haz 2};
    \ray(-2,1.5)(C2)(0.5)(1);
    \ray(C2)(C1)(0.5)(1);
    \ray(C1)(2,0)(1)(1);
    \planemirror[cyan,opacity=0.2](C1)(1)(-45)(0.1);
    \planemirror[cyan,opacity=0.2](C2)(1)(135)(0.1);
    \draw[dashed] (-2,0) -- (-0.5,0);
    \draw[latex-latex] (-1.5,0) -- (-1.5,1.5) node[left, pos=0.5] {$L$};

    \coordinate (C3)  at (1.5,4.5);
    \coordinate (C4)  at (1.5,3);
    \coordinate (C5)  at (0,3);
    \coordinate (C6)  at (0,6);
    \draw (-2,4.5) node[left] {Haz 1};
    \ray(-2,4.5)(C3)(0.4)(1);
    \ray(C3)(C4)(0.5)(1);
    \ray(C5)(C6)(0.75)(1);
    \ray(C4)(C5)(0.5)(-1);
    \ray(C6)(2,6)(1)(1);
    \planemirror[cyan,opacity=0.2](C3)(1)(135)(0.1);
    \planemirror[cyan,opacity=0.2](C4)(1)(45)(0.1);
    \planemirror[cyan,opacity=0.2](C5)(1)(-45)(0.1);
    \planemirror[cyan,opacity=0.2](C6)(1)(225)(0.1);
    \draw[dashed] (-2,6) -- (-0.5,6);
    \draw[dashed] (-2,3) -- (-0.5,3);
    \draw[dashed] (0,6.2) -- (0,7);
    \draw[dashed] (1.5,6.2) -- (1.5,7);
    \draw[latex-latex] (-1.5,3) -- (-1.5,4.5) node[left, pos=0.5] {$L$};
    \draw[latex-latex] (-1.5,4.5) -- (-1.5,6) node[left, pos=0.5] {$L$};
    \draw[latex-latex] (0,6.6) -- (1.5,6.6) node[above, pos=0.5] {$L$};
  \end{tikzpicture}
  \captionof{figure}{Problema \ref{p:PO201}\label{f:PO201}}
\end{center}
\end{minipage}
%
\begin{minipage}[c]{0.5\textwidth}
  \begin{center}
  \begin{tikzpicture}[line cap=round,line join=round]
  \planemirror[cyan,opacity=0.2](0,0)(6)(0)(0.2);
  \fill[gray,opacity=0.3] (2,0) -- (2,3.5) -- (2.1,3.5) -- (2.1,0) -- cycle;
  \draw (2,0) -- (2,3.5) node[above] {Pantalla};
  \coordinate (F)  at (-2,1.75);
  \draw[fill=gray] (F) circle(2.5pt) node[above left] {Fuente};
  \draw[latex-latex] (-2,0) -- (-2,1.7) node[left, pos=0.5] {$h$};
  \draw (0,-0.5) node[] {Espejo};
  \draw[latex-latex] (-2,-1) -- (2,-1) node[below, pos=0.5] {$s$};
  \draw (-2,-0.3) -- (-2,-1.2);
  \draw (2,-0.3) -- (2,-1.2);
  \coordinate (P)  at (2,2.7);
  \filldraw[] (P) circle(2pt) node[above right] {$P$};
  \draw (P) + (0.3,0) --++ (1,0);
  \draw[latex-latex] (P) + (0.7,0) --++ (0.7,-2.7) node[right, pos=0.5] {$y$};
  \ray(F)(P)(0.5)(1);
  \coordinate (x) at (-0.455,0);
  \ray(x)(P)(0.5)(1);
  \ray(F)(x)(0.5)(1);
  \draw (-0.3,2.5) node[] {1};
  \draw (0.8,0.9) node[] {2};
  \end{tikzpicture}
  \captionof{figure}{Problema \ref{p:PO203}\label{f:PO203}}
  \end{center}
\end{minipage}
%
\pma{\label{p:PO203}
Determine el desfasaje entre los rayos 1 y 2 que llegan al punto $P$ (ver figura~\ref{f:PO203}) como función de su altura $y$ respecto del espejo. El sistema está inmerso en un medio de índice $n$.\\
\rta{.87}{$\Delta\phi=\frac{2\pi n}{\lambda_0} \left( \sqrt{(y+h)^2+s^2}-\sqrt{(y-h)^2+s^2} \right) \pm \pi $}}


% \pma{\label{p:PO204}
% Determine el desfasaje entre los rayos 1 y 2 que llegan al punto $P$ en el dispositivo de la figura \ref{f:PO204}. El material que atraviesa el rayo 1 tiene un índice de refracción $n'$ y el sistema está inmerso en un medio de índice de refracción $n$.\\
% \rta{.75}{$\Delta\phi=\frac{2\pi}{\lambda_0}\left(n\sqrt{4h^2+s^2}-n(s-d)-n' d\right) \pm \pi$}}
%
% \begin{center}
%   \begin{tikzpicture}[line cap=round,line join=round]
%   \planemirror[cyan,opacity=0.2](0,0)(6)(0)(0.2);
%   \fill[gray,opacity=0.3] (2,0) -- (2,3.5) -- (2.1,3.5) -- (2.1,0) -- cycle;
%   \draw (2,0) -- (2,3.5) node[above] {Pantalla};
%   \coordinate (F)  at (-2,1.75);
%   \draw[fill=gray] (F) circle(2.5pt) node[above left] {Fuente};
%   \draw[latex-latex] (-2,0) -- (-2,1.7) node[left, pos=0.5] {$h$};
%   \draw (0,-0.5) node[] {Espejo};
%   \draw[latex-latex] (-2,-1) -- (2,-1) node[below, pos=0.5] {$s$};
%   \draw (-2,-0.3) -- (-2,-1.2);
%   \draw (2,-0.3) -- (2,-1.2);
%   \coordinate (P)  at (2,1.75);
%   \filldraw[] (P) circle(2pt) node[above right] {$P$};
%   \draw (P) + (0.3,0) --++ (1,0);
%   \draw[latex-latex] (P) + (0.7,0) --++ (0.7,-1.75) node[right, pos=0.5] {$h$};
%   \fill[gray,opacity=0.3] (-0.5,2.2) -- (0.5,2.2) -- (0.5,1.3) -- (-0.5,1.3) -- cycle;
%   \draw[] (-0.5,2.2) -- (0.5,2.2) -- (0.5,1.3) -- (-0.5,1.3) -- cycle;
%   \draw (0,1.75) node[] {$n'$};
%   \ray(F)(-0.5,1.75)(0.5)(1);
%   \ray(0.5,1.75)(P)(0.5)(1);

%   \coordinate (x) at (0,0);
%   \ray(x)(P)(0.5)(1);
%   \ray(F)(x)(0.5)(1);
%   \draw (1,2) node[] {1};
%   \draw (1,0.5) node[] {2};
%   \draw (-0.5,2.4) -- (-0.5,3);
%   \draw (0.5,2.4) -- (0.5,3);
%   \draw[latex-latex] (-0.5,2.8) -- (0.5,2.8) node[above, pos=0.5] {$d$};
% \end{tikzpicture}
% \captionof{figure}{Problema \ref{p:PO204}\label{f:PO204}}
% \end{center}




  \section{Experiencia de Young}
\rfigure
%
\pma{\label{p:P208}
Con el objetivo de determinar la longitud de onda de una fuente desconocida se realiza un experimento de interferencia de Young con una separación entre rendijas de $0.5$~mm y la pantalla situada a $1.75$~m. Sobre la pantalla se forman franjas brillantes consecutivas cuyos puntos medios distan $2.1$~mm. ¿Cuál es la longitud de onda de la luz utilizada?\\ \rta{.77}{$\lambda=600$~nm, en el medio en el que se realiza el experimento}}
%
\pma{
Se realiza el experimento de Young con luz de longitud de onda igual a $\SI{502}{\nano\metre}$. Se miden con cuidado las franjas sobre una pantalla que está a $\SI{1.20}{\metre}$ de la doble ranura, y se determina que el centro de la vigésima franja brillante está a $\SI{10.6}{\milli\metre}$ del centro de la franja brillante central. ¿Cuál es la separación entre las dos ranuras?\\
\rta{.95}{$\SI{1.14}{\milli\metre}$}}
%
\pma{
Dos ranuras muy angostas están separadas $\SI{1.80}{\micro\metre}$, colocadas a $\SI{35.0}{\centi\metre}$ de una pantalla y se iluminan con luz coherente de $\lambda = \SI{550}{\nano\metre}$. \textit{a}) ¿Cuál es la distancia entre la segunda y tercera líneas brillantes del patrón de interferencia? \textit{b}) ¿Cuál es la distancia entre la segunda y tercera líneas oscuras del patrón de interferencia? \textit{c}) Calcular nuevamente dichas distancias si el aparato completo (ranuras, pantalla y el espacio intermedio) se sumerge en agua, con un índice de refracción igual a $1.33$ para esta longitud de onda.\\
\rta{.95}{\textit{a}) $\SI{53.27}{\centi\metre}$; \textit{b}) $\SI{23.38}{\centi\metre}$; \textit{c}) $\SI{15.18}{\centi\metre}$ y $\SI{11.71}{\centi\metre}$}}
%
\pma{
Dos ranuras paralelas delgadas que están separadas $\SI{0.0116}{\milli\metre}$ son iluminadas por un rayo láser con longitud de onda de $\SI{585}{\nano\metre}$. \textit{a}) En una pantalla lejana muy grande, ¿cuál es el número total de franjas brillantes? (Sugerencia: Pregúntese cuál es el valor más grande que puede tener $\sin\theta$. ¿Qué le dice esto acerca de cuál es el valor máximo de $m$?). \textit{b}) ¿A qué ángulo con respecto a la dirección original del rayo se presentará la franja más distante de la franja brillante del centro?\\
\rta{.95}{\textit{a}) 39; \textit{b}) $\SI{73.37}{\degree}$}}
%
\pma{\label{p:P209}
En un experimento de doble rendija se observan franjas de interferencia utilizando luz de sodio ($\lambda_0= 589$~nm). ¿Cuál debería ser la longitud de onda utilizada para que la separación angular entre ellas sea 10\% mayor? Suponga válida la aproximación paraxial.\\
\rta{.95}{$\lambda_0 =647.9$~nm}}
%
\pma{\label{p:P213}
Un haz de luz monocromática incide perpendicularmente sobre cuatro rendijas muy estrechas e igualmente espaciadas. Si se cubren las dos rendijas centrales, el máximo de interferencia de cuarto orden se ve bajo el ángulo de 30º. ¿Bajo qué ángulo se ve el máximo de primer orden si se cubren las dos rendijas de los extremos, es decir, dejando las dos centrales abiertas?\\  \rta{.95}{$22.024$º}}
%
\pma{\label{p:P215}
Se hace incidir normalmente luz de longitud de onda $\lambda=632.8$~nm (en el vacío) procedente de un láser de helio-neón sobre un plano que contiene dos rendijas. El primer máximo de interferencia se encuentra a 8~cm del máximo central, cuando se observa el patrón de interferencia en una pantalla de 1~m de ancho (centrada en el máximo de orden 0) situada a 2~m de distancia de las rendijas. \textit{a}) Calcule la separación entre las rendijas. \textit{b}) ¿Cuántos máximos de interferencia se observan en la pantalla? \textit{c}) ¿Qué máximos se observan en la pantalla si la luz incide con un ángulo de 8º sobre las rendijas? \textit{d}) Lo mismo que en \textit{c}), pero con la zona entre las rendijas y la pantalla llena de agua ($n_{agua} = 4/3$).\\ \rta{.95}{\textit{a}) $15.84\,\mu$m ($15.82\,\mu$m usando aproximación paraxial); \textit{b}) 13; \textit{c}) 12 máximos, desde $m=\pm 2$ hasta $m=\mp 9$; \textit{d}) 16 máximos, de $m=\pm 4$ a $m=\mp 11$}}
%
\pma{
Suponga que usted ilumina dos ranuras delgadas con luz coherente monocromática en el aire, y determina que produce su primer mínimo de interferencia en $\SI{35.20}{\degree}$ a ambos lados del punto brillante central. Luego sumerge las ranuras en un líquido transparente, las ilumina con la misma luz y determina que el primer mínimo ahora ocurre en $\SI{19.46}{\degree}$. ¿Cuál es el índice de refracción de este líquido?\\
\rta{.95}{$1.73$}}
%
\pma{\label{p:P212}
A través de dos ranuras angostas separadas por una distancia de $\SI{0.300}{\milli\metre}$ pasa luz coherente que contiene dos longitudes de onda, $\SI{660}{\nano\metre}$ (rojo) y $\SI{470}{\nano\metre}$ (azul), y se observa el patrón de interferencia en una pantalla colocada a $\SI{5.00}{\metre}$ de las ranuras. ¿Cuál es la distancia en la pantalla entre las franjas brillantes de primer orden para las dos longitudes de onda?\\
\rta{.95}{$\SI{3.17}{\milli\metre}$}}
%


  \section{Difracción}
\rfigure
% \setcounter{equation}{0}
%
\pma{\label{p:P231}
Considere un haz de ondas planas que incide sobre una placa que contiene una rendija. Si la intensidad del máximo central en el patrón de difracción es $I_\text{max}$, determine la intensidad (relativa a la máxima) de los primeros 3 máximos secundarios de difracción.\\
\rta{.95}{$I_1=0.047\,I_\text{max}$, $I_2=0.016\,I_\text{max}$, $I_3=0.008\,I_\text{max}$}
}
%
\pma{\label{p:P237}
Los rayos paralelos de un haz de luz de longitud de onda $\lambda_0=650$~nm pasan por una pantalla en la cual se halla una rendija de $0.5$~mm de ancho y 3~cm de alto. El patrón de difracción generado por esta rendija se observa en una pantalla ubicada 50~cm a continuación de la rendija. Calcular la distancia, medida sobre la pantalla, entre el máximo principal del patrón de difracción y \textit{a}) los primeros mínimos, \textit{b}) los primeros máximos secundarios.\\
\rta{.95}{\textit{a}) $\pm0.65$~mm; \textit{b}) $\pm0.93$~mm}}
%
\pma{\label{p:P238}
Se observa el patrón de difracción generado por una rendija de $0.25$~mm de ancho sobre una pantalla situada a una distancia de $2.5$~m de ella. La campana principal de difracción tiene un ancho de 11~mm. ¿Cuál es la longitud de onda de la luz utilizada?\\
\rta{.95}{550~nm}
}
%
\pma{\label{p:P239}
Se ilumina con luz infrarroja proveniente de un láser de He-Ne ($\lambda_0= 1152.2$~nm en el vacío) una pantalla que contiene una rendija estrecha, y se determina que el centro de la décima banda oscura en el patrón de difracción de Fraunhofer se encuentra en un ángulo de $\SI{6.3}{\degree}$ respecto del eje central. \textit{a}) Calcule el ancho de la rendija. \textit{b}) ¿Para qué ángulo aparecerá la décima franja oscura si todo el arreglo se sumerge en agua $\left(n_a(\lambda_0) = 1,326\right)$ en lugar de aire?\\ \rta{.95}{\textit{a}) $a=\SI{105}{\micro\meter}$; \textit{b}) $\theta_{10}=\SI{4.747}{\degree}$}}
%
\pma{\label{p:P240}
Calcule el ancho $a$ de la ranura rectangular que producirá en su patrón de difracción de campo lejano una campana principal con un ancho angular $\delta\theta$ de \textit{i}) $\SI{30}{\degree}$, \textit{ii}) $\SI{45}{\degree}$, \textit{iii}) $\SI{90}{\degree}$, \textit{iv}) $\SI{180}{\degree}$, al ser iluminada con de longitud de onda 550~nm.\\
\rta{.95}{\textit{i}) $\SI{2.125}{\micro\meter}$; \textit{ii}) $\SI{1.437}{\micro\meter}$; \textit{iii}) $\SI{0.778}{\micro\meter}$; \textit{iv}) $\SI{0.550}{\micro\meter}$}}
%
\pma{\label{p:P241}
El patrón de difracción producido por una única rendija, cuando se la ilumina con luz de 550~nm, es observado sobre una pantalla ubicada a 40~cm a continuación de la placa que contiene a la rendija. Si la distancia entre el primer y quinto mínimo es de $0.4$~mm, determinar el ancho de la rendija.
\rta{.95}{\textit{a}) $a=2.2$~mm}}
%
\begin{minipage}[t]{.5\textwidth}
  \textbf{Pantallas complementarias}. Consideremos dos superficies $\Sigma$ y $\Sigma'$ de manera tal que en los puntos donde la superficie $\Sigma$ es transparente, $\Sigma'$ es opaca y viceversa, como se muestra en la figura \ref{f:babinet}. Por ejemplo, si $\Sigma$ es un agujero circular en una pantalla opaca, entonces $\Sigma'$ es un disco opaco del mismo tamaño y posición en una pantalla transparente. Este tipo de pantallas se denominan complementarias. Se puede mostrar que el patrón de difracción de la pantalla $\Sigma$ es el mismo que produce la pantalla $\Sigma'$ (salvo en la dirección de la onda incidente). Este hecho se conoce con el nombre de \textit{principio} o \textit{teorema de Babinet}.
\end{minipage}
\hfill
%\begin{figure}
\begin{minipage}[t]{.55\textwidth}
\strut\vspace*{-\baselineskip}

  \pgfdeclarelayer{layer1}
  \pgfdeclarelayer{layer2}
  \pgfdeclarelayer{layer3}
  \pgfsetlayers{main, layer3, layer2, layer1}

  \begin{center}
  \begin{tikzpicture}[x={(1cm,0.4cm)}, y={(8mm, -3mm)}, z={(0cm,1cm)}, line cap=round, line join=round, scale=0.75]
    \begin{pgfonlayer}{layer1}
      \object{3}
      \ray(0,-1,0)(2.5,-1,0)(0.5)(1);
      \ray(0,1,0)(2.5,1,0)(0.5)(1);
    \end{pgfonlayer}
    \begin{pgfonlayer}{layer3}
      \image{8}
    \end{pgfonlayer}
    \begin{pgfonlayer}{layer2}
      \lens{5.4}
    \end{pgfonlayer}

    \draw (3,0,-2) node[] {Objeto 1};
    \draw (5.4,0,-1.5) node[] {Lente};
    \draw (8,0,-2) node[] {Pantalla 1};

    \begin{pgfonlayer}{layer1}
      \begin{scope}[shift={(0,0,-4.5)}]
        \objecttriangle{3};
        \ray(0,-1,0)(2.5,-1,0)(0.5)(1);
        \ray(0,1,0)(2.5,1,0)(0.5)(1);
      \end{scope}
    \end{pgfonlayer}
    \begin{pgfonlayer}{layer3}
      \begin{scope}[shift={(0,0,-4.5)}]
        \image{8}
        \lens{5.4}
      \end{scope}
    \end{pgfonlayer}

    \begin{scope}[shift={(0,0,-4.5)}]
      \draw (3,0,-2) node[] {Objeto 2};
      \draw (5.4,0,-1.5) node[] {Lente};
      \draw (8,0,-2) node[] {Pantalla 2};
    \end{scope}
  \end{tikzpicture}
  \captionof{figure}{Pantallas complementarias.}\label{f:babinet}
  \end{center}
\end{minipage}
%
\pma{\label{p:P234}
La luz de un láser de He-Ne ($\lambda_0=632.8$~nm) se dirige hacia un cabello humano, en un experimento para medir su diámetro examinando el patrón de difracción. El cabello está montado sobre un bastidor y el patrón de difracción se observa sobre una pantalla que se encuentra a $0.75$~m del mismo. Si el ancho de la campana principal es de $1.46$~cm, ¿cuál es el diámetro del cabello?\\
\rta{.95}{$\SI{65}{\micro\meter}$}}
%
\pma{\label{p:P242}
¿Cuántos máximos de interferencia estarán contenidos dentro de la campana principal de difracción en el diagrama de interferencia-difracción de \textbf{dos rendijas}, si la separación $d$ entre ellas es 5 veces su ancho $a$? ¿Y si es $6.5$ veces?
\rta{.95}{9; 13}}
%
\pma{\label{p:P243}
Se observa un diagrama de interferencia-difracción de Fraunhofer producido al iluminar con luz de longitud de onda $\lambda = 500$~nm dos rendijas de ancho $a$ separadas por $0.1$~mm. \textit{a}) Calcule el ancho $a$ de las rendijas si el décimo máximo de interferencia se produce en el mismo ángulo que el segundo mínimo de difracción. \textit{b}) En ese caso, ¿cuántas franjas brillantes se verán en la primera campana secundaria de difracción? ¿A qué órdenes corresponden?\\
\rta{.95}{\textit{a}) $a=0.02$~mm; \textit{b}) 4 ($m=6, 7, 8$ y $9$)}}
%
\pma{\label{p:P244}
Un haz de luz de 550~nm de longitud de onda ilumina dos rendijas de ancho $0.025$~mm y separación $0.15$~mm. \textit{a}) ¿Cuántos máximos de interferencia entran dentro de la campana principal de difracción? \textit{b}) ¿Cuál es el cociente entre la intensidad del tercer máximo de interferencia a un lado de la línea central y la intensidad de este máximo central?\\
\rta{.95}{\textit{a}) El central más 5 máximos a cada lado, un total de 11; \textit{b}) $\frac{4}{\pi^2}=0.4053$}}
%
\pma{\label{p:P245}
La luz de proveniente de una fuente de longitud de onda $\lambda$=475 nm pasa a través de una doble rendija, produciendo un patrón de interferencia-difracción cuyo gráfico de intensidad $I(\theta)$ se puede ver en la figura \ref{f:P245}. \textit{a}) Calcule el ancho de las ranuras y la separación entre ellas. \textit{b}) Verifique que las intensidades de los máximos de interferencia para $m=1$ y $m=2$ son las correctas. \textit{c}) ¿Cuál es el valor de $I_0$?\\
\rta{.95}{\textit{a}) $a=5.45\,\mu$m y $d=21,8\,\mu$m; \textit{b}) los valores calculados para el primer y segundo máximo: $I_1=5.674$~mW/cm\textsuperscript{2} y $I_2=2.837$~mW/cm\textsuperscript{2} respectivamente, y los leídos del gráfico: $5.67\pm0.07$~mW/cm\textsuperscript{2} y $2.89\pm0.07$~mW/cm\textsuperscript{2}, no muestran diferencias significativas entre sí; \textit{c}) $I=1.75$~mW/cm\textsuperscript{2}}}
%
\begin{center}
  \begin{tikzpicture}[scale=1]
    \begin{axis}[
                 every major x tick/.append style={thick,blue},
                 clip=false,
                 grid=both,
                 minor x tick num=1,        %un minor tick es decir 0.5
                 minor y tick num=1,
                 xmin=0, xmax=15,           %min y max para los ejes, NO PARA EL DOMINIO
                 ymin=0, ymax=8,
                 %axis y line=center,        %alinea el eje al centro de la figura
                 %axis x line=middle,        %sino pone 2 ejes x
                 xtick  align=center,
                 xlabel={Ángulo~[grados]},
                 ylabel={Intensidad~[$\si{\milli\watt/\metre^2}$]},
                 width=12cm,
                 height=8cm
                ];
    \addplot [color=green!70!black, very thick] [samples= 400, domain=0.001:15]  {7*(cos(180*sin(x)*4/sin(5))*sin(180*sin(x)*1/sin(5))/(pi*sin(x)*1/sin(5)))^2};
    \end{axis}
  \end{tikzpicture}
  \captionof{figure}{Problema \ref{p:P245}\label{f:P245}}
\end{center}
%
\pma{\label{p:P267}
Un haz de luz de longitud de onda (en el vacío) $\lambda_0= 500$~nm incide sobre una placa con dos rendijas, separadas entre sí una distancia $d = 25 \mu$m y de ancho $a = \frac15 d$. La placa se encuentra en la división entre dos medios de índice de refracción $n_1 = 1.20$ y $n_2 = 1.28$, respectivamente. Al hacer incidir el haz con un ángulo $\theta_i$, se observa que el patrón de intensidades se desplaza 30 franjas respecto de la situación con incidencia normal. La pantalla de observación está ubicada a $1$~m de distancia de la placa y está centrada respecto del eje del sistema. \textit{a}) Calcule el ángulo de incidencia $\theta_i$. \textit{b}) ¿A qué distancia del eje del sistema se encuentra el centro de la campana principal de difracción? \textit{c}) Calcule la irradiancia en el centro de la pantalla.\\
\rta{.95}{\textit{a}) $\theta_i=30$º; \textit{b}) $53$~cm; \textit{c}) $I=0$}}
%
\pma{\label{p:P305}
Un haz de luz de longitud de onda (en el vacío) $\lambda_0= 600$~nm incide con un ángulo de $0.967$º sobre una placa que contiene dos rendijas de $20\,\mu$m de ancho y separadas $80\,\mu$m.  Se observa el patrón de interferencia en una pantalla situada a 1 m de la placa con las rendijas. \textit{a}) ¿Cuánto mide la interfranja?
\textit{b}) Calcule la intensidad que se registra en el centro de la pantalla, sabiendo que desde cada rendija sale luz con $5$~W/m\textsuperscript{2} de intensidad.\\
\rta{.95}{\textit{a}) $7.5$~mm; \textit{b}) $3.08$~W/m\textsuperscript{2}}}
  % \section{Polarización}
\rfigure
%
\pma{\label{p:P174}
Una onda luminosa que se propaga a lo largo del eje $z$ incide sobre un polarizador lineal, contenido siempre en el plano $xy$. Si la intensidad de la luz que llega al polarizador es $I_0$, discuta cómo varía (o no) la intensidad luminosa transmitida al girar el eje de transmisión, en los casos en que la luz incidente sea:
\bemca
 \item circularmente polarizada dextrógira; %a
 \item circularmente polarizada levógira;  %b
 \item elípticamente polarizada dextrógira; %c
 \item elípticamente polarizada levógira; %d
 \item linealmente polarizada; %e
 \item luz natural. %f
\eemca
\noindent
\rta{.95}{\textit{a}),\textit{ b}),\textit{ f}) La intensidad no varía y vale $\frac12 I_0$; \textit{c}),\textit{ d}) la intensidad varía entre $I_{min}\neq 0$ e $I_{max}$ (con $I_{max} + I_{min} = I_0$); \textit{e}) la intensidad varía entre 0 e $I_0$}
}
%
\pma{\label{p:P175}
Un haz de luz no polarizada de irradiancia $I_0$ pasa a través de una secuencia de dos polarizadores lineales ideales. ¿Cuál debe ser la orientación relativa entre sus ejes de transmisión, si el haz emergente tiene una irradiancia de \textit{a}) $I_0/2$; \textit{b}) $I_0/4$?\\
\rta{.95}{Sus ejes de transmisión deben estar \textit{a}) paralelos; \textit{b}) a 45º}}
%
\pma{\label{p:PO176}
Dos láminas polarizadoras tienen sus ejes de transmisión cruzados perpendicularmente con $\theta_1=0$ y $\theta_3=\frac{\pi}{2}$ (ver figura \ref{f:PO176}). Entre ellas se inserta una tercera lámina de modo que su eje de transmisión forme un ángulo $\theta_2$ con el de la primera lámina. Si sobre la primer lámina incide luz natural, demuestre que la intensidad transmitida a través de las tres láminas es máxima cuando $\theta_2=45$º.}
%
\pma{\label{p:PO177}
La lámina polarizadora intermedia del problema \ref{p:PO176} está girando alrededor del eje $z$ con velocidad angular $\omega=4\pi$~rad/s. Si sobre la primera incide luz natural de intensidad 8~W/m$^2$, determine los valores máximo y mínimo de la intensidad luminosa transmitida a través de las tres láminas y cada cuánto tiempo se observarán los máximos de intensidad. \\
\rta{.95}{$I_{max}=1$~W/m$^2$, $I_{min}=0$~W/m$^2$; los máximos en la irradiancia se observan cada $0.125$~s}}
%
\pma{\label{p:PO178}
Un haz de luz no polarizada se envía a través del sistema de tres láminas polarizadoras del problema \ref{p:PO176}, cuyos ejes de transmisión forman ángulos $\theta_1=40$º, $\theta_2=20$º, $\theta_3=40$º. ¿Qué porcentaje de la luz que llega al primer polarizador sale por el tercero?\\
\rta{.75}{$3.125$~\%}}

\pgfdeclarelayer{layer1}
\pgfdeclarelayer{layer2}
\pgfdeclarelayer{layer3}
\pgfsetlayers{main, layer3, layer2, layer1}

\begin{center}
\begin{tikzpicture}[x={(1cm,0.4cm)}, y={(8mm, -3mm)}, z={(0cm,1cm)}, line cap=round, line join=round, scale=0.75]
  % \begin{pgfonlayer}{layer1}
  %   \object{3}
  %   \ray(0,-1,0)(2.5,-1,0)(0.5)(1);
  %   \ray(0,1,0)(2.5,1,0)(0.5)(1);
  % \end{pgfonlayer}
  % \begin{pgfonlayer}{layer3}
  %   \image{8}
  % \end{pgfonlayer}
  \begin{pgfonlayer}{layer1}
    \lens{2}
    \begin{scope}[canvas is yz plane at x=2]
      \draw[dashed] (0.7,-0.8) -- (-1.4,1.6);
      \draw (0,1.2) -- (0,1.8);
      \draw[-latex] (0,1.4) arc (90:133:1.4) node[above, pos=0.5] {$\theta_3$};
    \end{scope}
    \ray(0,0,0)(2,0,0)(0.5)(-1);

  \end{pgfonlayer}

  \begin{pgfonlayer}{layer2}
    \lens{5}
    \begin{scope}[canvas is yz plane at x=5]
      \draw[dashed] (0.7,0.8) -- (-1.4,-1.6);
      \draw (0,-1.2) -- (0,-1.8);
      \draw[-latex] (0,-1.4) arc (270:227:1.4) node[below, pos=0.5] {$\theta_2$};
    \end{scope}
    \ray(5,0,0)(2,0,0)(0.5)(-1);
  \end{pgfonlayer}

  \begin{pgfonlayer}{layer3}
    \ray(11,0,0)(8,0,0)(0.5)(-1);
    \lens{8}
    \begin{scope}[canvas is yz plane at x=8]
      \draw[dashed] (0.7,-0.8) -- (-1.4,1.6);
      \draw (0,1.2) -- (0,1.8);
      \draw[-latex] (0,1.4) arc (90:133:1.4) node[above, pos=0.5] {$\theta_1$};
    \end{scope}
    \ray(8,0,0)(5,0,0)(0.5)(-1);
  \end{pgfonlayer}

  \draw[-latex] (10,0,0) -- (10,0,1) node[right] {$y$};
  \draw[-latex] (10,0,0) -- (10,1,0) node[right] {$x$};
  
\end{tikzpicture}
\captionof{figure}{Problemas \ref{p:PO176}, \ref{p:PO177}, \ref{p:PO178}.}\label{f:PO176}
\end{center}
%
\pma{\label{p:P179}
Se desea girar 90º el plano de polarización de un haz de luz polarizado linealmente de intensidad $I_0$. ¿Cómo podría hacerse usando únicamente polarizadores lineales? Diseñe un experimento donde la pérdida de la intensidad total sea menor al 40\%. Justifique analíticamente su respuesta.\\
\rta{.95}{Usando $N$ polarizadores, de manera tal que cada uno tenga su eje de transmisión rotado $\pi/2N$ respecto del anterior. A la salida se obtiene un haz de intensidad $I=I_0\cos^{2N}\left(\frac{\pi}{2N}\right)$ y una rotación total de $\pi/2$ respecto del haz inicial. Se verifica que $I/I_0>0.6$ para $N\geq5$}}

  \section{Preguntas sobre óptica para el análisis}
\rfigure
\textit{En esta sección se requiere que se brinden respuestas argumentadas.}

\pma{
Se realiza un experimento de interferencia con dos ranuras, y las franjas se proyectan en una pantalla. Después, todo el aparato se sumerge agua, ¿cómo cambia el patrón de las franjas?
}
%
\pma{
A través de dos ranuras delgadas pasa luz monocromática coherente que se ve en una pantalla alejada. ¿Las franjas brillantes en la pantalla se encontrarán igualmente separadas? Si es así, ¿por qué? Si no, ¿cuáles están más cerca de estar igualmente separadas?
}
%
\pma{
En un patrón de interferencia de dos ranuras sobre una pantalla distante, ¿las franjas brillantes están a la mitad de la distancia que hay entre las franjas oscuras?
}
%
\pma{
Las luces de un automóvil distante, ¿formarían un patrón de interferencia de dos fuentes?
}
%
\pma{
Se iluminan con luz coherente de longitud de onda $\lambda$ dos ranuras estrechas separadas por una distancia $d$. Si $d$ es menor que cierto valor mínimo, no se observan franjas oscuras. Explique lo que sucede. En términos de $\lambda$, indique cuál es este valor mínimo de $d$.
}
%
\pma{
¿Por qué podemos observar fácilmente los efectos de la difracción en el caso de las ondas sonoras y de las ondas en el agua, pero no en el caso de la luz?
}
%
\pma{
A través de una sola ranura de ancho $a$ pasa luz de longitud de onda $\lambda$ y frecuencia $f$. Se observa el patrón de difracción en una pantalla a una distancia $S$ de la ranura. De las acciones siguientes, ¿cuáles reducen la anchura del máximo central? \textit{a}) Disminuir el ancho $a$ de la ranura. \textit{b}) Disminuir la frecuencia $f$ de la luz. \textit{c}) Disminuir la longitud de onda $\lambda$ de la luz. \textit{d}) Disminuir la distancia $S$ de la ranura a la pantalla.
}
%
\pma{
En un experimento de difracción que utiliza ondas con longitud de onda $\lambda$, no habrá mínimos de intensidad (es decir, no habrá franjas oscuras) si la anchura de la rendija es lo suficientemente pequeña. ¿Cuál es el ancho máximo de rendija con el cual ocurre esto?
}
%
% \pma{
% Cuando la luz no polarizada incide en dos polarizadores cruzados, no se transmite luz. Un estudiante afirmó que si se insertaba un tercer polarizador entre los otros dos, habría algo de transmisión. ¿Tiene sentido esto? ¿Cómo podría un tercer filtro incrementar la transmisión?
% }
%



  \end{document}