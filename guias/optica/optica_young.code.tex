\section{Experiencia de Young}
\rfigure
%
\pma{\label{p:P208}
Con el objetivo de determinar la longitud de onda de una fuente desconocida se realiza un experimento de interferencia de Young con una separación entre rendijas de $0.5$~mm y la pantalla situada a $1.75$~m. Sobre la pantalla se forman franjas brillantes consecutivas cuyos puntos medios distan $2.1$~mm. ¿Cuál es la longitud de onda de la luz utilizada?\\ \rta{.77}{$\lambda=600$~nm, en el medio en el que se realiza el experimento}}
%
\pma{
Se realiza el experimento de Young con luz de longitud de onda igual a $\SI{502}{\nano\metre}$. Se miden con cuidado las franjas sobre una pantalla que está a $\SI{1.20}{\metre}$ de la doble ranura, y se determina que el centro de la vigésima franja brillante está a $\SI{10.6}{\milli\metre}$ del centro de la franja brillante central. ¿Cuál es la separación entre las dos ranuras?\\
\rta{.95}{$\SI{1.14}{\milli\metre}$}}
%
\pma{
Dos ranuras muy angostas están separadas $\SI{1.80}{\micro\metre}$, colocadas a $\SI{35.0}{\centi\metre}$ de una pantalla y se iluminan con luz coherente de $\lambda = \SI{550}{\nano\metre}$. \textit{a}) ¿Cuál es la distancia entre la segunda y tercera líneas brillantes del patrón de interferencia? \textit{b}) ¿Cuál es la distancia entre la segunda y tercera líneas oscuras del patrón de interferencia? \textit{c}) Calcular nuevamente dichas distancias si el aparato completo (ranuras, pantalla y el espacio intermedio) se sumerge en agua, con un índice de refracción igual a $1.33$ para esta longitud de onda.\\
\rta{.95}{\textit{a}) $\SI{53.27}{\centi\metre}$; \textit{b}) $\SI{23.38}{\centi\metre}$; \textit{c}) $\SI{15.18}{\centi\metre}$ y $\SI{11.71}{\centi\metre}$}}
%
\pma{
Dos ranuras paralelas delgadas que están separadas $\SI{0.0116}{\milli\metre}$ son iluminadas por un rayo láser con longitud de onda de $\SI{585}{\nano\metre}$. \textit{a}) En una pantalla lejana muy grande, ¿cuál es el número total de franjas brillantes? (Sugerencia: Pregúntese cuál es el valor más grande que puede tener $\sin\theta$. ¿Qué le dice esto acerca de cuál es el valor máximo de $m$?). \textit{b}) ¿A qué ángulo con respecto a la dirección original del rayo se presentará la franja más distante de la franja brillante del centro?\\
\rta{.95}{\textit{a}) 39; \textit{b}) $\SI{73.37}{\degree}$}}
%
\pma{\label{p:P209}
En un experimento de doble rendija se observan franjas de interferencia utilizando luz de sodio ($\lambda_0= 589$~nm). ¿Cuál debería ser la longitud de onda utilizada para que la separación angular entre ellas sea 10\% mayor? Suponga válida la aproximación paraxial.\\
\rta{.95}{$\lambda_0 =647.9$~nm}}
%
\pma{\label{p:P213}
Un haz de luz monocromática incide perpendicularmente sobre cuatro rendijas muy estrechas e igualmente espaciadas. Si se cubren las dos rendijas centrales, el máximo de interferencia de cuarto orden se ve bajo el ángulo de 30º. ¿Bajo qué ángulo se ve el máximo de primer orden si se cubren las dos rendijas de los extremos, es decir, dejando las dos centrales abiertas?\\  \rta{.95}{$22.024$º}}
%
\pma{\label{p:P215}
Se hace incidir normalmente luz de longitud de onda $\lambda=632.8$~nm (en el vacío) procedente de un láser de helio-neón sobre un plano que contiene dos rendijas. El primer máximo de interferencia se encuentra a 8~cm del máximo central, cuando se observa el patrón de interferencia en una pantalla de 1~m de ancho (centrada en el máximo de orden 0) situada a 2~m de distancia de las rendijas. \textit{a}) Calcule la separación entre las rendijas. \textit{b}) ¿Cuántos máximos de interferencia se observan en la pantalla? \textit{c}) ¿Qué máximos se observan en la pantalla si la luz incide con un ángulo de 8º sobre las rendijas? \textit{d}) Lo mismo que en \textit{c}), pero con la zona entre las rendijas y la pantalla llena de agua ($n_{agua} = 4/3$).\\ \rta{.95}{\textit{a}) $15.84\,\mu$m ($15.82\,\mu$m usando aproximación paraxial); \textit{b}) 13; \textit{c}) 12 máximos, desde $m=\pm 2$ hasta $m=\mp 9$; \textit{d}) 16 máximos, de $m=\pm 4$ a $m=\mp 11$}}
%
\pma{
Suponga que usted ilumina dos ranuras delgadas con luz coherente monocromática en el aire, y determina que produce su primer mínimo de interferencia en $\SI{35.20}{\degree}$ a ambos lados del punto brillante central. Luego sumerge las ranuras en un líquido transparente, las ilumina con la misma luz y determina que el primer mínimo ahora ocurre en $\SI{19.46}{\degree}$. ¿Cuál es el índice de refracción de este líquido?\\
\rta{.95}{$1.73$}}
%
\pma{\label{p:P212}
A través de dos ranuras angostas separadas por una distancia de $\SI{0.300}{\milli\metre}$ pasa luz coherente que contiene dos longitudes de onda, $\SI{660}{\nano\metre}$ (rojo) y $\SI{470}{\nano\metre}$ (azul), y se observa el patrón de interferencia en una pantalla colocada a $\SI{5.00}{\metre}$ de las ranuras. ¿Cuál es la distancia en la pantalla entre las franjas brillantes de primer orden para las dos longitudes de onda?\\
\rta{.95}{$\SI{3.17}{\milli\metre}$}}
%

